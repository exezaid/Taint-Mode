
Privacy has been an ever important issue in today's society. We enter our 
credit card numbers on webs to do on-line shopping. We use our email to
subscribe on-line magazines, newsletters, and advertisements which intrigue us.
We have detail personal information recorded in bank companies and medical
organizations. Various kinds of computing systems managing tons of
data for us alleviate our already busy life a lot, especially in a
information explosion age like today. But what happened if a computing 
system have some flaws in processing our private data and, unintentionally
or intentionally, reveal them to bad guys? We probably receive extra bill from
a credit card company, be overwhelmed by spam, or get a strange call during night.
An unsafe computing system generally lead to a bigger problem. Would it 
be great if we have a mechanism to control data dissemination in
a system? It will substantially increase reliability of a system.

In many cases, security problems regarding improper handling sensitive data 
arise from unintentional mistakes of programmers.
Consider a salary program of a bank system where there are two procedures, 
perProc is for normal personnel to access nonsensitive data, such as personal 
salary, and admProc is for manager to access sensitive data, salary of all
personnel, for instance. It is possible for a programmer to call the wrong
procedure by mistake when improving the code, in which admProc is called 
when the user is just a normal staff. If he or she doesn't realize
such flaw and update the running program, now all personnel can see the 
salary of his or her colleagues.

There are also some cases that somebody try to leak information
deliberately. Nowadays computer programs 
are usually huge and complex. It is normal that a person who we don't even
know writes a small extension and hope us to merge it into our program. 
It may require some sensitive data to perform the task. But the problem is
how do we know his or her program use these sensitive data properly. He or she
could send a copy of these data to somebody who will use them illegally. 
On the other hand, examining the code to find the usages of sensitive data
manually are too tedious and almost impossible if the extension is big.

The problems above arise from lack of security mechanisms to define and enforce
policy of use of sensitive data in a program, which is one type of information-flow 
security. As described in~\cite{Sabelfeld:Myers:JSAC},
there has been many conventional security mechanisms used in practical 
computing system for a long time. These mechanisms includes access controls, 
firewalls, and anti-virus software. But none of them are end-to-end security 
mechanisms. Access control, for instance, only guarantees a user has sufficient 
privilege to acquire secret data, but not how the secret data is propagated 
and used. To guarantee information-flow security, we need some mechanism to
keep track of data flowing in a program and make sure they are not used
improperly.

In the past few decades, researchers has shown that language techniques are 
very useful in information-flow security~\cite{Sabelfeld:Myers:JSAC}. Static
program analysis is discovered to be effective~\cite{Denning:Denning:Certification}
because it has similar nature as information-flow security. Both of
them are reasoning on all branches of program at the same time, rather than
certain execution path of program. The compositional property of 
information-flow security makes it suitable to enforce security policy as
type checking at compile time. There are two full-fledged programming language
implementations so far, Jif~\cite{jif} and FlowCaml~\cite{FlowCaml}. 
They both adopt programming language techniques to specify and enforce 
information-flow policy.

Li and Zdancewic~\cite{Li:Zdancewic:CSFW}, on the other hand, developed an 
embedded language based on the {\em arrows} interface in Haskell to enforce 
information-flow policy. We call it FlowHaskell.
Without information-flow type system directly implemented in compiler
of Haskell, static constraints are used to guarantee information-flow policy. 
If all the constraints are satisfied, then underlying secure computation is 
well-typed according to an information-flow policy.
Information-flow constraints are collected and resolved at run-time before 
secure computation can execute. This approach has a major advantage.  
That is, Programmers can switch between host language and embedded 
language easily. They can get most of jobs done in their familiar
language and focus on security sensitive part with embedded language.
In this way, programmers would be more likely to use information-flow
language in practice. Many existing applications can also be upgraded
to guarantee information-flow security with small changes, comparing to 
rewriting whole program.

In addition to sequential programs, multithreaded programs are very 
prevalent in today's computing system as well. Network programming
is one of the areas where multithreading is used and encouraged.
A new thread is often spawned to handle a new client after a connection
is established. In many applications, clients demand limited resources
from a server. Those resources can be accessed by client threads in
any order. Indeed, share resources are usually required in concurrent
programming with producer-consumer pattern.
However, due to share resource, multithreaded programs exploit new
information-flow security problems and become an interesting 
research topic of its own~\cite{Sabelfeld:Myers:JSAC}. There have
been many theoretical results but only few or maybe no practical
implementation at all. Jif~\cite{jif}, FlowCaml~\cite{FlowCaml},
and FlowHaskell~\cite{Li:Zdancewic:CSFW} all
have no support for multithreaded information-flow security.

\paragraph{Contribution}
Our project is based on FlowHaskell~\cite{Li:Zdancewic:CSFW}, 
and intend to support information-flow security in multithreaded programming. 
Our work includes,
\begin{enumerate}
\item Defining a new framework for complex security types
\item Developing primitives for reference manipulation in the new framework
\item Extending support for information-flow security in multithreaded 
programming
\item Evaluating expressiveness and usefulness of the library by case study 
\end{enumerate}

The following sections explain background knowledge
necessary to understand rest of the report.
