
\section{Information Flow in Multithreaded Program}
Share memory plays an essential role in the collaboration of 
threads in multithreaded programs, but, at the same time, it also
introduce new information channel. Non-determinism of execution
order is the source of the problem. Consider the following
two thread commands:
\begin{align*}
b_1:&~(\ifthenelse{\cod{h}~>~0}{\cod{sleep}(120)}{\cod{sleep}(0)});~\cod{l}~:=~1 \\
b_2:&~\cod{sleep}(60);~\cod{l}~:=~0
\end{align*}
where $\cod{h}$ and $\cod{l}$ are a high and a low variable 
respectively. The content of $\cod{l}$ could be either 1 or 0,
depending on the execution order of last commands of $b_1$ and $b_2$.
Assume $\cod{sleep}(n)$ is to execute n consecutive $skip$ commands, and 
each $skip$ is seen as one step for scheduler. If we have a scheduler 
reschedules every 80 steps, the program above will leak information
with a high probability. When $b_1$ is chosen to run first and $\cod{h}$
is greater than 0, in the middle of $\cod{sleep}(120)$, $b_2$ will be
scheduled to run and finish within 80 steps and then $b_1$ is scheduled
again and finish. The content of $\cod{l}$ is 1. But if $\cod{h}$ is less
than 0, $b_1$ can finish in 80 steps and then $b_2$ runs. In this case,
$\cod{l}$ would be 0. An attacker can, therefore, deduce the value of $\cod{h}$ 
by the low result $\cod{l}$ with certain confidence. If an attacker can
run the program many times, he or she can have a even higher confidence 
based on statistical results. Notice that both $b_1$ and $b_2$ are secure
under the notion of {\it noninterference} discussed in the last section.
As a result, we need new security policies in multithreaded programs.

\subsection{Noninterference in Multithreaded Programs}
{\em Possibilistic noninterference}, explored by Smith and 
Volpano~\cite{Smith:Volpano:MultiThreaded}, states that the possible 
{\it low} outputs of a program is independent of {\it high} inputs. 
Both $b_1$ and $b_2$ obey the definition because whatever $\cod{h}$ is, 
the possible results of $\cod{l}$ are \{0,1\}. This notion of 
noninterference assumes an attacker has no ability to infer information
from probabilistic results of a program. On the other hand, 
{\em probabilistic noninterference} has a more stringent 
restriction~\cite{Volpano:Smith:Probabilistic}. It requires not only the 
possible appearances but also the probability distribution of {\it low} outputs
being the same when {\it high} inputs vary. 

\subsection{Existing Approaches}
Russo and Sabelfeld~\cite{Russo:Sabelfeld:CSFW06} separate threads of 
different security levels and treat them differently in scheduler 
to ensure the interleaving of publicly-observable events does not 
depend on sensitive data. Threads are categorized as high threads 
and low threads, and are put into different threadpools. The running time 
of high threads is invisible to low threads. They introduce primitives such 
as $\cod{hide}$, $\cod{unhide}$, and $\cod{hfork}$ to signal scheduler 
to move a thread to the corresponding part of a threadpool. This
approach requires interaction between threads and scheduler.

Volpano and Smith~\cite{Volpano:Smith:Probabilistic} propose a language
with static threads. They assume an uniform scheduler in which
each thread has the same probability being chosen in thread pool, and also
introduce a new primitive $\cod{protect}$ to force a sequence of commands
executing atomically. They claim if every total command with a guard containing
a {\it high} variable executes atomically, the program exhibits {\em probabilistic
noninterference}. But protected commands cannot diverge, otherwise, it will
freeze all other threads as well. The following is an example of using $\cod{protect}$:
\begin{align*}
c_1:&~\cod{protect}(\ifthenelse{\cod{h}~>~0}{\cod{sleep}(120)}{\cod{sleep}(0)});~\cod{l}~:=~1 \\
c_2:&~\cod{sleep}(60);~\cod{l}~:=~0
\end{align*}
Since the if-statement of $c_1$ executed atomically, $c_2$ cannot
distinguish the timing difference between both branches of the if-statement.
The probability of the result of $\cod{l}$ is thus independent of $\cod{h}$.
One problem about $\cod{protect}$ is that its semantic is nonstandard and
not clear how to be implemented. A strictly uniform scheduler is hard to 
achieve in practice as well.

Boudol and Castellani~\cite{Castellani:Boudol:TCS02} give a type system which 
reject programs with low assignments after a high guard command. Since every
low assignment is not dependent on running time of high value, low outputs 
will not change as high inputs vary. They have proved {\em possibilistic 
noninterference} for well-typed programs. However, the following program
will be rejected:
\begin{align*}
d_1:&~(\ifthenelse{\cod{h}~>~0}{\cod{sleep}(120)}{\cod{sleep}(0)});~\cod{l_a}~:=~1 \\
d_2:&~\cod{sleep}(60);~\cod{l_b}~:=~0
\end{align*}
where $l_a$ and $l_b$ are low variables.
This shows the type system has potential to reject secure and useful programs.
The approaches mentioned so far deal with {\em internal timing} leaks, which
assumes an attacker can only judge timing difference by looking at
low outputs.

Sabelfeld and Sands~\cite{Sabelfeld:Sands:CSFW00} provide a padding
technique to make both branches of a high guard command have identical
running time. This approach can deal with {\em external timing}, which 
assumes an attacker has the power to count exact running time of a 
program. Nevertheless, it requires underlying operating system 
and hardware to preserve timing property of each command. Padding techniques
may also change the efficiency of a program.

Information flow in multithreaded program is still an open question.
All the researches above gives theoretical results.
There is no example of multithreaded information-flow language that is implemented 
by any information-flow security compiler so far.
