
\section{Information Flow}
Information-flow problem is about how to keep track of data of different 
categories flowing in a program and enforce a policy. A policy
will meet a certain security goal. In confidentiality aspect of
a computing system, data are categorized by different sensitivity. 
Some variables may contain higher sensitivity data and thus 
have higher sensitivity than other variables. A mechanism is required
to guarantee there is no information leak in a program, which means
no sensitive data is revealing to a user who has no privilege
to acknowledge the data. In the rest of the report, we only
address confidentiality.

\subsection{Noninterference}
{\it Noninterference} is a security policy to judge if confidentiality
is preserved in a program. Assume we split input data and variables
into {\it high} and {\it low}. {\it Noninterference}
states that {\it low} part of result(or output) should be 
independent of {\it high} input, and an attacker can only
look at variables in {\it low} part. That is, when we change any
{\it high} input, the output which can be observed by an attacker 
should still be the same.

\subsection{Types of Flow}
There are generally two types of flow that can leak information. 
The first type is
{\it explicit} flow which information is leaked by passing sensitive
data from a sensitive variable directly to a non-sensitive variable.
Assume we have two variables, called $\cod{h}$ and $\cod{l}$. 
$\cod{h}$ contains sensitive data and $\cod{l}$ is non-sensitive
and can be read by an attacker.
A simple assignment $\cod{l} := \cod{h}$ will leak information.
The other way is via {\it implicit} 
flow~\cite{Denning:Denning:Certification}. Consider the following
program:
\begin{align*}
a_1:~\ifthenelse{\cod{h}~>~0}{\cod{l}~:=~1}{\cod{l}~:=~0}
\end{align*}
Now the content of $\cod{l}$ is dependent on the content of $\cod{h}$,
and an attacker can have information about $\cod{h}$ by looking at
$\cod{l}$. Program $a_1$ fails
to comply with {\it noninterference} because if we assign $\cod{h}$ 
with different values, $\cod{l}$ may have different results.

\subsection{Termination}
There are two kinds of security specifications related to
{\it noninterference}, termination-insensitive and
termination-sensitive. In termination-insensitive,
{\it noninterference} states if a program runs with two
different memories that only differ in {\it high} 
part and both executions terminate, the {\it low} part
of resulting memories should be identical. Nevertheless, in
termination-sensitive, {\it noninterference} states that
if a program runs with two different memories that
only differ in {\it high} part, then it guarantee
two cases. If both executions terminate, the {\it low} part
of resulting memories is identical. Or both executions 
diverge(infinite loop).

Which security specification is good enough depends on
the capability of an attacker. If an attacker can only
observe the result of execution within a program, 
he will never observe an diverging execution. Thus,
termination-insensitive is strong enough. But if an
attacker can observe the result from outside of the
program, he can judge an execution diverges after
certain period of time elapsed. Therefore, only
termination-sensitive is good enough.

\subsection{Information Downgrading}
In an information-flow secured language, at some point, 
we may need to release sensitive information on purpose.
For example, we would like to release the information that if a user
has entered a correct password. Some kind of declassification
mechanism is necessary to prevent a situation where information
is well protected but cannot be used when there is a need. 
Both~\cite{jif}and~\cite{Li:Zdancewic:CSFW}~has declassification 
mechanism.
