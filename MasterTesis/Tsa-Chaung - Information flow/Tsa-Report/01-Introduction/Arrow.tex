
\section{The Arrows Interface}
Arrows is proposed by Hughes~\cite{Hughes:SCP00} to represent a set of computations. 
It has similar purposes to monads as a common interface for libraries.
Comparing to monads, arrows have input type parameterised and have the ability to 
choose computations based on static information. There
are many type classes in arrows and each type class provides a different functionality.
The following introduces several type classes that will be used in the report.
The fundamental type class is $\cod{Arrow}$:
\begin{verbatim}
class Arrow a where
  pure :: (b -> c) -> a b c
  (>>>) :: a b c -> a c d -> a b d
  first :: a b c -> a (b, d) (c, d)
  second :: a b c -> a (d, b) (d, c)
  (***) :: a b c -> a b' c' -> a (b, b') (c, c')
  (&&&) :: a b c -> a b c' -> a b (c, c')
\end{verbatim}
where $\cod{a}$ is an abstract data type for arrows ,and $\cod{b}$ and $\cod{c}$ are types
of input and output respectively. Operator $\cod{pure}$ lifts a function of type 
($\cod{b}->\cod{c}$) into an arrow. Infix operator \arrowop{>} composes two arrows in
sequence where output type of first arrow should be identical to input type of second arrow.
Operator $\cod{first}$ composes an arrow parallel with an identity arrow, pure id. These three
operators are minimum to define an $\cod{Arrow}$ instance and the remaining operators have 
default definition based on the three. Notice that some arrows need to implement all operators
because the intended meanings are different from default. Operation $\cod{second}$
has an opposite effect of parallel composition to $\cod{first}$. Operator \arrowop{*} parallel
composes two arbitrary arrows, and \arrowop{\&} duplicates the input and feed them to both arrows.

The basic $\cod{Arrow}$ type class does not support conditional computations. Type class
$\cod{ArrowChoice}$ provide such ability via $\cod{Either}$ type in Haskell.
\begin{verbatim}
data Either a b = Left a | Right b

class Arrow a => ArrowChoice a where
  left :: a b c -> a (Either b d) (Either c d)
  right :: a b c -> a (Either d b) (Either d c)
  (+++) :: a b c -> a b' c' -> a (Either b b') (Either c c')
  (|||) :: a b d -> a c d -> a (Either b c) d
\end{verbatim}
Operator $\cod{left}$ transforms an arrow to support conditional computations based on 
type constructor of $\cod{Either}$. If input is of constructor $\cod{Left}$, value inside 
$\cod{Left}$ will be fed to arrow in $\cod{left}$. Otherwise, input will pass through as 
output. Other operators have default definitions based on $\cod{left}$. 
Combinator $\cod{right}$ feeds input to the arrow inside if the input is
of constructor $\cod{Right}$. Similarly, \arrowop{+} and $|||$ feed input to one of
arrows depending on the type constructor of input.

Another useful type class is $\cod{ArrowLoop}$ which provides the ability to define 
computation with feedback.
\begin{verbatim}
class Arrow a => ArrowLoop a where
  loop :: a (b,d) (c,d) -> a b c
\end{verbatim}
Combinator $\cod{loop}$ takes part of output, which is type of $\cod{d}$, and feed it 
as part of input. The whole computation only done once.
% Should we have a figure showing ArrowLoop computation?
% Should I give examples of using arrow interface?

Due to the structure of {\em arrows}, a point-free style of programming is natural and 
straightforward. But in some cases, it is a little bit tedious to use point-free style.
One instance is when a lot of intermediate values are required, we need to use nested
parallel combinator, such as \arrowop{\&}, to pass them along
computation. Paterson~\cite{Paterson:ICFP01} proposed a $\cod{do}$-like notation for arrows.
Programmers write pointed program in arrow notations, and compilers transform them to 
primitive arrow combinators before compiling.

Both~\cite{Hughes:ISSAFP05} and~\cite{Paterson:TFP03} are good tutorials for further 
study of arrows.

