
\section{Pure Problem}
\label{chap3:pure}
% explain why pure introduce such problem but not other primitives
% explain why Flat in FlowArrow cannot work 
% explain definition of new pure.
Combinator \co{pure} lifts a Haskell function into \co{FlowArrowRef}. 
It serves as a basic building block in FlowHaskellRef because it is the only way to
utilise Haskell functions in the security language. Other combinators provided in
FlowHaskellRef only deal with some simple computations, such as sequencing two computations.
The programmers need \co{pure} to process data with Haskell functions.

However, there are two difficulties in deciding the output security label of combinator \co{pure}.
One difficulty arises from the output type of a Haskell function is not fixed, and neither is the output
security label. The other problem is the dependency of inputs and outputs of a Haskell function is not clear.
It is not like in other primitives. For instance, the output security label of \co{createRef} is
($\st~\mathbf{ref}^\ell$) where $\st$ is the same as input security label and $\ell$ is provided by
the programmers. Both the output security label and the dependency with the input security label
are known. Considering the following program.
\begin{verbatim}
 pure (\(a,b) -> if a then (a,False) else (b,a))
\end{verbatim}
Assume \co{a} is a \co{Low} variable and \co{b} is a \co{High} variable. Depending on the value of \co{a},
the output security label can be (\co{Low},\co{Low}) or (\co{High},\co{Low}). A Haskell function may
contain more complex expressions and makes it impossible to deduce the dependency between
the inputs and the outputs if the function body is not revealed in advance.

In FlowHaskell, a constructor \co{Flat} is used to represent the security flow of a \co{pure} computation. Since no 
information downgrading can happen inside a Haskell function, output has a security label as high as
security label of input. Moreover, only one lattice label is used to represent security label for all
data type. No output type and dependency between inputs and outputs are required.
That is why \co{Flat} is an adequate choice in the previous work. 
But in FlowHaskellRef, different data types are protected by different security types. The input and output
security label may at the same security level but with different security type constructors. 
For example, the input is of type $int$ and the output is of type ($int,int$). The input security label
is of the form $\ell$, while the output security label is of the form ($\ell,\ell$).
Therefore, adopting \co{Flat} becomes inadequate in FlowHaskellRef.

Our solution of this problem is to approximate the output security label of \co{pure}.
As showed in Figure~\ref{fig:flowarrowref:typesystem0}, the output security label of \co{pure} is
$\mathbf{high}$. It means everything becomes top secret afterward. By this definition, we dodge the
two difficulties mentioned above. However, it severely restricts the usefulness of \co{pure}. 
Consider a case that the programmers want to increase the content of a reference with security label
($\cod{low}~\mathbf{ref}^{\cod{low}}$) by 1, which is illustrated in the following program.
\begin{verbatim}
 ... >>>
 (idRef 
  &&& 
  (readRef (SecLabel (Lab Low)) >>>
   pure (\x -> x+1))) >>>
 writeRef
\end{verbatim}
The value \co{x+1} becomes $\mathbf{high}$ after \co{pure} and can not be written back to the reference
again. To mitigate the limitation, a new combinator \co{lowerA} is introduced and explained in
Sec.~\ref{chap3:lower}.

 
