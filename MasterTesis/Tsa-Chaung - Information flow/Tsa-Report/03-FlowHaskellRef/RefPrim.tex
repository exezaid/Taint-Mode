
\section{Reference Primitive}
\label{chap4:reference}
% intro SecRef and operations
% intro RefMonad
% type system of reference primitive

\begin{figure}[t]
  \[\begin{array}{c}

    \inference[$CREATEREF$]{}
              { e(\st),\emptyset,\emptyset\proves \cod{createRef}~\ell~:~
              \res{\typ}{\st}~->~\res{\typ~ref}{\st~\mathbf{ref}^{\ell}}} \\ \\

    \inference[$READREF$]{}
              { \top,\{\ell\guard \sts{2}\},\{\sts{1}\sqsubseteq \sts{2}\}
                \proves \cod{readRef}~\sts{2}~:~
                \res{\typ~ref}{\sts{1}~\mathbf{ref}^{\ell}}~->~\res{\typ}{\sts{2}}} \\ \\

    \inference[$WRITEREF$]{}
              { e(\st),\{\ell\guard \st\},\emptyset
                \proves \cod{writeRef} :
                \res{(\typ~ref,\typ)}{(\st~\mathbf{ref}^{\ell},\st)}~->~\res{()}{\bot}} \\
    \end{array}
  \]
\caption{Type system of reference primitives}
\label{fig:reference:typesystem}
\end{figure}

FlowHaskellRef adopts mutable references in the IO monad for experiment. An abstract data type
\co{SecRef} is introduced as following(in RefOp.hs):
\begin{verbatim}
data SecRef a = MkSecRef (IORef a) (a -> a)
\end{verbatim}
Data type \co{SecRef} parameterises on the type of a reference's content. The constructor
\co{MkSecRef} takes an \co{IORef} and a {\em projection} function. 
The projection function is applied to the result of reading the reference and returns
a value of the same type. The purpose of this function is explained in Sec.~\ref{chap3:lower}.

Three standard operations for \co{SecRef} is provided(in RefOp.hs).
\begin{verbatim}
newSecRef :: a -> IO (SecRef a)
newSecRef a = do r <- newIORef a
                 return (MkSecRef r id)

readSecRef :: SecRef a -> IO a
readSecRef (MkSecRef r f) = do a <- readIORef r
                               return (f a)

writeSecRef :: SecRef a -> a -> IO ()
writeSecRef (MkSecRef r f) a = writeIORef r a
\end{verbatim}
The projection function is set to \co{id} when creating new \co{SecRef}. In \co{readSecRef},
the projection function \co{f} is applied to the result of reading internal IORef, \co{r}.
Function \co{writeSecRef} simply writes a value to a reference.

Type class \co{RefMonad} specifies a common interface for any monads and corresponding references
which can be used in FlowArrowHaskell. The definition is as following(in RefOp.hs):
\begin{verbatim}
class RefMonad m r | m -> r where
  createMRef :: a -> m (r a)
  readMRef :: (r a) -> m a
  writeMRef :: (r a) -> a -> m ()
\end{verbatim}
Type class \co{RefMonad} parameterises on a monad, \co{m}, and a reference type in the monad, \co{r}.
It provides standard operations for reference manipulation. 
Instance (\co{RefMonad} \co{IO} \co{SecRef}) is implemented directly by each operation of
\co{SecRef}.

Three standard operations for reference manipulation, called \co{createRef}, \co{readRef}, 
and \co{writeRef}, are implemented in FlowHaskellRef. They lift methods of \co{RefMonad} into 
\co{FlowArrowRef}. The implementation follows the type system in 
Figure~\ref{fig:reference:typesystem} directly and can be found in FlowArrowRef.hs.

