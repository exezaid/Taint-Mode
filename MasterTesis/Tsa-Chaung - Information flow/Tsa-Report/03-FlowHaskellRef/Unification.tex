
\section{Unification}
\label{chap3:unification}

\newcommand{\unify}[1]{\stackrel{{#1}}{\sim}}
\newcommand{\ct}{c^l}
% Note
% Motivation
% Show UType and explain
% Show unification part of FlowArrowRef
% - how uniset is produced(find vars in flow and uniset, and rename, and then replace)
% Regulation when using unification variables to define new primitive

\begin{figure}[t]
\[
\alpha ::= \omega\ |\ \ell
\]
\caption{Extended lattice type}
\label{fig:unif:lt}
\end{figure}

\begin{figure}[t]
\[
\st\ ::=\ \alpha\ |\ \st\ \mathbf{ref}^{\alpha}\ |\ (\st,\st)\ |\ (\mathbf{either}\ \st\ \st)^{\alpha}\ 
        |\ \mathbf{high}\ |\ t
\]
\caption{Extended security type}
\label{fig:unif:st}
\end{figure}

\begin{figure}[t]
\[
\ell\unify{\emptyset} \ell\quad \quad \omega\unify{(\omega:=\alpha)} \alpha
\]
\caption{Unification of lattice}
\label{fig:unif:ul}
\end{figure}

\begin{figure}[t]
\[
  \begin{array}{c}
  \inference[]{\alpha_1\unify{U} \alpha_2}{\alpha_1\unify{U} \alpha_2} \quad
  
  \inference[]{\alpha_1\unify{U_1} \alpha_2 \quad U_1\sts{1}\unify{U_2} U_1\sts{2}}
              {\sts{1}\ \mathbf{ref}^{\alpha_1}\unify{U_2.U_1} \sts{2}\ \mathbf{ref}^{\alpha_2}} \quad

  \inference[]{\sts{1}\unify{U_1} \sts{3} \quad U_1\sts{2}\unify{U_2} U_1\sts{4}}
              {(\sts{1},\sts{2})\unify{U_2.U_1} (\sts{3},\sts{4}) }     \\ \\

  \inference[]{\alpha_1\unify{U_1} \alpha_2 \quad
               U_1\sts{1}\unify{U_2} U_1\sts{3} \quad {U_2}U_1\sts{2}\unify{U_3} {U_2}U_1\sts{4}}
              {(\mathbf{either}\ \sts{1}\ \sts{2})^{\alpha_1}\unify{U_3.U_2.U_1} 
               (\mathbf{either}\ \sts{3}\ \sts{4})^{\alpha_2}} \quad

  \inference[]{}
              {\mathbf{high}\unify{\emptyset} \mathbf{high}} \\ \\
  
  \inference[]{t \not\in \cod{FV}(\st)}
              {t\unify{t:=\st} \st}  \quad

  \inference[]{}
              {\mathbf{high}\unify{\emptyset} \top}  \quad

  \inference[]{}
              {\mathbf{high}\unify{\omega:=\top} \omega}  \\ \\

%  \inference[]{High\unify{U_1} \alpha \quad U_1High\unify{U_2} U_1\st}
%              {High\unify{U_2.U_1} \st\ ref^{\alpha} } \\ \\
%
  \inference[]{\mathbf{high}\unify{U_1} \sts{1} \quad U_1\mathbf{high}\unify{U_2} U_1\sts{2}}
              {\mathbf{high}\unify{U_2.U_1} (\sts{1},\sts{2})}  \quad

  \inference[]{\mathbf{high}\unify{U_1} \alpha \quad
               \mathbf{high}\unify{U_2} U_1\sts{1} \quad {U_2}U_1\mathbf{high}\unify{U_3} {U_2}U_1\sts{2}}
              {\mathbf{high}\unify{U_3.U_2.U_1} (\mathbf{either}\ \sts{1}\ \sts{2})^{\alpha}} \\ \\
  \end{array}
\]
\caption{Unification of security type}
\label{fig:unif:us}
\end{figure}

\begin{figure}[t]
\[
\ct ::= \ct\sqcap \ct\ |\ \ct\sqcup \ct\ |\ \mathbf{extr}\ \ct\ |\ \mathbf{fst}\ \ct\ |\ \mathbf{snd}\ \ct\        
        |\ \tagup\ l\ \ct\ |\ \decl\ l\ \ct\ |\ \st
\]
\caption{Unification constraint type}
\label{fig:unif:ct}
\end{figure}

\begin{figure}[t]
\[
 \begin{array}{c}
 \inference[]{\sts{1}\unify{U} \sts{2}}
             {\cod{Id}\ \sts{1}\unify{U} \cod{Id}\ \sts{2}} \quad
% \inference[]{|[\ct|]=\cod{Id}\ \st}
%              {v\unify{(v:=\st)} \ct}  \quad
 \inference[]{|[\ct|]\unify{U} \beta}
             {\ct\unify{U} \beta} \\ \\
 \end{array}
\]
\caption{Unification constraint elimination}
\label{fig:unif:ce}
\end{figure}

Primitive \co{createRef} takes a value and returns a reference, and the output security label 
, $\st~\mathbf{ref}^{\ell}$, contains two parts. 
One is the security label of the content and the other is the security label
of its own identity. The security label of the content is identical to that of the input
security label according to the type system in Figure~\ref{fig:reference:typesystem}. But how 
should the input security label be determined? In the framework
of FlowHaskell, the only feasible way is to ask the programmers to specify the security labels.
The following is an example using this approach:
\begin{verbatim}
pure (\_ -> 0) >>> createRef (SecLabel HIGH) LOW >>> 
readRef (SecLabel HIGH) LOW >>> ...
\end{verbatim}
In \co{createRef}, the programmers are required to specify the security label of the input as well as
the security label of the identity. Similar in \co{readRef}, both the security labels of the content and
the identity need to be specified. In many cases, however, the creation and reading of a reference reside in
very different parts of a program, and the programmers have to remember the security labels. This is
not only tedious but error-prone. Consider a small error in \co{readRef} as following:
\begin{verbatim}
pure (\_ -> 0) >>> createRef (SecLabel HIGH) LOW >>>
readRef (SecLabel LOW) LOW >>> ...
\end{verbatim}
The \co{HIGH} content of the reference becomes \co{LOW} due to a mistake by specifying wrong security label. To avoid this kind of mistakes, a new mechanism to pass security labels through computation 
is demanded. Unification is adopted in FlowHaskellRef to resolve the limitation.

The main idea is to extend the ability of passing complex security label in the \co{flow} field 
of \co{FlowArrow}. Security label variables are supported to define primitives and aimed to
receive security labels from previous computations when combined with arrow combinators.
When certifying a computation, all the variables are unified first and the computation is 
certified as normal.

\subsection{Unification Types}
Two kinds of unification variables are required. One can be unified with a lattice type and the other
is used to  unify with a complex security type. In Figure~\ref{fig:unif:lt} a new constructor, variable $\omega$,
is extended in the original lattice type. It is implemented by a new data type \co{Label} as following
(in Lattice.hs):
\begin{verbatim}
data Label l = Lab l | LVar String
\end{verbatim}
To unify with a complex security type, $\st$ has a new constructor, $t$, for unification variable, as in 
Figure~\ref{fig:unif:st}. Data type \co{SecType} is modified to reflect the new design(in Lattice.hs).
\begin{verbatim}
data SecType l = SecLabel (Label l) 
               | Ref (SecType l) (Label l)
               | SecVar String
               | ...
\end{verbatim}
Constructors not showed above is the same.

The unification semantics for lattice types is depicted in Figure~\ref{fig:unif:ul}. When two lattice labels
unify with each other, they should be identical. If one is a variable, a new unification
result is generated. Figure~\ref{fig:unif:us} shows the unification semantics for $\st$. 
Symbol $\unify{U}$ denotes a unification that produces some unification results, $U$. A unification
result along with a security type, $U\st$, updates the security type, $\st$, according to the unification
result, $U$. If more than one unification result are produced, the final unification result is the 
composition of all results in the same order as they are generated. Notice that there is no rule to
unify a $\mathbf{high}$ and a $\st~\mathbf{ref}^\alpha$. The reason is the security label of the content
of a reference is invariant in the sub-typing relation in Figure~\ref{fig:subtyping}. Unifying with
$\mathbf{high}$ may break the relation.

Many operators for complex security labels are necessary to generate constraints in \co{FlowArrowRef}, but
those operators may only defined for complex security labels which contain no unification variables. 
Therefore, those operation should be retained and performed until all variables inside the corresponding complex
security labels are resolved. A new type called unification constraint, as in Figure~\ref{fig:unif:ct}, 
represents those operators and postpones operations until all variables involved are unified to a value.
Those constraints are implemented by data type \co{U}(in Unification.hs):
\begin{verbatim}
data U s = Meet (U s) (U s)
         | Join (U s) (U s)
         | LExtr (U s)
         | Fst (U s)
         | Snd (U s)
         | Tag (U s) (U s)
         | Decl (U s) (U s) (U s)
         | Id s
\end{verbatim}
Type variable \co{s} is \co{SecType} in our case. 
Constraints $\sqcap$ and $\sqcup$ correspond to the \co{Meet} and the \co{Join} constructors. 
The operations are performed in method \co{mlabel\_meet} and \co{mlabel\_join} of \co{SecType} respectively. 
Constraint $\mathbf{extr}$ is implemented by \co{LExtr} and resolved by method \co{mextract}.
Constraint $\mathbf{fst}$ and $\mathbf{snd}$ returns the first and the second element of a pair respectively.
Constraint $\tagup$ is implemented by constructor \co{Tag} and performed by method \co{mlabel\_tag}.
Constraint $\decl$ corresponds to constructor \co{Decl} and resolved by method \co{mlabel\_decl}. One more
parameter in \co{Decl} than those in constraint $\decl$ is due to implementation issues about \co{SecHigh}. 
It is a \co{SecType} with expected label annotations and returned as a result when \co{SecHigh} is encountered.
The last constraint is just a \co{SecType} and implemented by constructor \co{Id}.
Besides, any \co{Label} is wrapped in a \co{SecLabel} constructor so both data type \co{SecType} and data type 
\co{Label} can share a constraint type. For example, a label (\co{Lab} \co{l}) becomes 
(\co{SecLabel} (\co{Lab} \co{l})).

The unification semantics for unification constraint are showed in Figure~\ref{fig:unif:ce}. When two constraints
are both \co{SecType}s, they are unified according to the unification semantics of $\st$. The second rule says
if a constraint contains no variable, the corresponding operation is performed before unification, 
denoted by symbol $|[|]$. Otherwise, the constraint is put back to the unification set and try again later.

\subsection{Unification in FlowArrowRef}
\label{chap3:unification:flowarrowref}
% show real definition of FlowArrow and FlowArrowRef
% explain new Flow definition
% explain usage of uniset
% explain alpha conversion
% explain binding operators with unification
To extend unification in FlowHaskellRef, some modification of original \co{FlowArrow}
is required. The following is the new definition of \co{FlowArrow}(in FlowArrowRef.hs):
\begin{verbatim}
data FlowArrow l a b c = FA
    { computation  :: ((SecType l) -> (SecType l)) 
                      -> a b c
    , flow         :: Flow (SecType l)
    , constraints  :: [Constraint (SecType l)]
    , gconstraints :: [GConstraint (SecType l)]
    , uniset       :: [(U (SecType l), U (SecType l))]
    }
data Flow l = Trans l l 
\end{verbatim}
Field \co{computation} becomes a function that accepts a substitution function and returns a underlying
computation. A substitution function takes a \co{SecType} and returns a new \co{SecType} with some 
label annotations replaced.
The reason of introducing a substitution function is explained in Sec.~\ref{chap3:lower}.
New definition of \co{Flow} removes constructor \co{Flat} which loses constructor information of a complex
security label. A new field, \co{uniset}, contains a list of pairs of constraint types which are aimed 
to be unified.

The following is the definition of \co{FlowArrowRef}(in FlowArrowRef.hs):
\begin{verbatim}
data FlowArrowRef l a b c = 
  FARef { flowarrow :: FlowArrow l a b c
        , pc :: (SecType l)
        , constraintsRef :: [Constraint (SecType l)]
        , gconstraintsRef :: [UGConstraint (SecType l)]
        }
\end{verbatim}
This is similar to the conceptual definition in Sec.~\ref{chap3:flowarrowref:concept} except that 
the type of \co{pc} becomes \co{SecType} rather than a lattice type. This is only to have a uniform
type in \co{uniset}. The interpretation of \co{pc} remains the same.

When defining new primitives with unification, variables normally appear in the input security label
because it is the place to acquire information from previous computations.
Those variables can be used to define constraints within the 
same \co{FlowArrowRef}. But when combining two \co{FlowArrowRef} computations via arrow combinators,
it is possible that two computations have variables with identical names but are actually distinct.
Those variables are likely to be inconsistent and cannot be resolved to a single value in unification 
algorithm. As a result, every variable is renamed to a distinct name in arrow combinators. 
To collect all variables in a \co{FlowArrowRef} computation, it is sufficient to collect variables in the field
\co{flow} and \co{uniset}. New variables must either in field \co{flow} or field
\co{uniset} so that they can be unified to a value afterward. 
After all variables of the arrows in the arguments of an arrow combinator are collected, they are renamed with
unique names. Then, variables in all fields are replaced with their new names.

Since security labels are passed through field \co{flow},
combinators need to decide new \co{flow}s that pass 
security labels as well as fulfill the type system described in Sec.~\ref{chap3:flowarrowref:typesystem}.
Combinator \co{pure} defines its \co{flow} to be ($t~->~\mathbf{high}$) which implements the type system
directly(in FlowArrowRef.hs).
\begin{verbatim}
pure f = FA { ...
              flow = Trans (SecVar "x0") SecHigh
              ... }
\end{verbatim}
The \co{flow} of combinator (\arrowop{>}) is defined by the following function(in FlowArrowRef.hs):
\begin{verbatim}
flow_seq :: Lattice l => 
            Flow (SecType l) -> Flow (SecType l)
            -> ( Flow (SecType l)
               , [Constraint (SecType l)]
               , [(U (SecType l),U (SecType l))])
flow_seq (Trans s1 s2) (Trans s3 s4)=
    if (hasVarSecType s3)
      then (Trans s1 s4, [],[(Id s2,Id s3)])
      else (Trans s1 s4, [LEQ s2 s3],[])
\end{verbatim}
Given two \co{flow}s of sequenced \co{FlowArrowRef}s, \co{flow\_seq} returns new \co{flow}, required 
constraints, and additional unification pairs. If \co{s3} has no variable, this is 
the case depicted in type system, and a new constraint $\cod{s2}\sqsubseteq \cod{s3}$ is generated. 
On the other hand, if \co{s3} contains variables, it is aimed to receive information from \co{s2}, so \co{s3}
should be unified with \co{s2}. This means $\cod{s2}=\cod{s3}$ and the constraint 
$\cod{s2}\sqsubseteq \cod{s3}$ is satisfied implicitly.

For combinators \co{first}, \co{second}, and (\arrowop{*}), new flows are decided by \co{flow\_pair}
(in FlowArrowRef.hs). 
\begin{verbatim}
flow_pair :: Lattice l => Flow (SecType l) 
             -> Flow (SecType l) -> Flow (SecType l)
flow_pair (Trans s1 s2) (Trans s3 s4) = 
                Trans (Pair s1 s3) (Pair s2 s4)
\end{verbatim}
It simply puts input and output security labels in pair constructors respectively.
For combinator (\arrowop{\&}), more analyses are required 
in \co{flow\_diverge}(in FlowArrowRef.hs).
\begin{verbatim}
flow_diverge :: Lattice l => 
                Flow (SecType l) -> Flow (SecType l)
                -> ( Flow (SecType l)
                   , [Constraint (SecType l)] 
                   , [(U (SecType l), U (SecType l))])
flow_diverge (Trans s1 s2) (Trans s1' s2') =
  if hasVarSecType s1
    then if hasVarSecType s1'
           then (Trans s1 out_flow, [], [(Id s1,Id s1')])
           else (Trans s1 out_flow, [LEQ s1 s1'], [])
    else if hasVarSecType s1'
           then (Trans s1' out_flow, [LEQ s1' s1], [])
           else (Trans (SecVar "x0") out_flow
                , [], [(Id (SecVar "x0")
                       ,Meet (Id s1) (Id s1'))])
  where
  out_flow = (Pair s2 s2')
\end{verbatim}
If both in-flow security labels of combined \co{FlowArrowRef} contain variables, since they are both expected 
to be the same as the out-flow security label of previous computations, they should be identical. Thus, any of 
them can represent new in-flow security label and a unification pair is generated to force them to be the same.
If only one of in-flow security labels contains variables, the one with variables is regarded as new 
in-flow security label because it is expected to receive security labels from previous computations.
A constraint is required to guarantee that the other in-flow security label adheres to
sequence rule. That is, security level of program flows cannot downgrade in sequential combination. 
This is captured by constraint \co{LEQ}, which is equivalent to $\sqsubseteq$. 
If both of them have no variables, the type system is implemented directly.

Function \co{flow\_either} defines
flow for \co{left}, \co{right}, and (\arrowop{+})(in FlowArrowRef.hs).
\begin{verbatim}
flow_either (Trans s1 s2) (Trans s3 s4) = 
               Trans (SecEither s1 s3 (LVar "x0")) 
                     (SecEither s2 s4 (LVar "x0"))
\end{verbatim}
A lattice label for the identity of \co{SecEither} is received from previous computations and passed
as the label of output \co{SecEither}. It implements the type system directly.

Function \co{flow\_converge} returns flow of ($|||$), a list of constraints, and a list of unification
pairs(in FlowArrowRef.hs):
\begin{verbatim}
flow_converge :: Lattice l => 
              Flow (SecType l) -> Flow (SecType l)
              -> ( Flow (SecType l)
                 , [GConstraint (SecType l)]
                 , [(U (SecType l),U (SecType l))])
flow_converge (Trans s1 s2) (Trans s1' s2') =
  let l = (LVar "x0") in
  let s_out = (SecVar "x1") in
  (Trans (SecEither s1 s1' l) s_out
  ,[GUARD (SecLabel l) s1, GUARD (SecLabel l) s2]
  ,[(Id s_out, Join (Id s2) (Id s2'))])
\end{verbatim}
The identity label of input security label \co{SecEither} is a variable to receive a lattice label
from previous computations.
The resulting output security label is the join of two output security labels.
Combinator ($|||$) takes out the components of a \co{SecEither} and does computation, 
so two \co{GUARD} constraints are required to protect the contents. 
They denote $\cod{l}\guard \cod{s1}$ and $\cod{l}\guard \cod{s2}$.

%Method \co{loop} in \co{ArrowLoop} takes an {\em arrow} whose input and output are both pairs, and 
%returns a new {\em arrow}. The new input is the first element of the input, and the new output is
%the first element of the output. The second element of the output is taken as the second
%element of the input to form a feedback computation.
%Function \co{flow\_loop} takes the \co{flow} of the input {\em arrow} and returns a new \co{flow} 
%and an additional constraint.(in FlowArrowRef.hs)
%\begin{verbatim}
%flow_loop :: Lattice l => Flow (UType l)
%             -> (Flow (UType l)
%                ,[Constraint (UType l)])
%flow_loop (Trans a b) = 
%    (Trans (UFst a) (UFst b)
%    ,[LEQ (USnd b) (USnd a)]) 
%\end{verbatim}
%New \co{flow} is a flow from the first element of the input security label to the first element of the
%output security label. A new constraint is required because the feedback introduced in \co{loop}.

Except field \co{flow}, other fields of \co{FlowArrowRef} implement the type system 
in Sec.~\ref{chap3:flowarrowref:typesystem} directly, although the enforcement of the type system are 
postponed until unification variables being unified successfully.

\subsection{Defining New Primitives}
% no new variables has 'a' as first char.
% new variables only introduced in flow field.
New primitives of \co{FlowArrowRef} can be introduced based on unification scheme.
Unification variables specified in primitives provide a possibility of receiving information from 
other computations.
However, there are two rules which should be strictly obeyed to guarantee success of unification
algorithm.
\begin{enumerate}
\item Names of new variables cannot begin with a character 'a' and followed by a number.
\item New variables must be defined either in input security labels or unification set.
\end{enumerate}
First rule comes from the fact that variables are renamed to the format internally. A new variable may 
have a name collision with existing ones if not named properly. Second rule guarantees all variables
is unified.
