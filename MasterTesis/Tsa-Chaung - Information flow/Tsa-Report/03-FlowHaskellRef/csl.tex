
\section{Complex Security Types}
\label{chap3:CSL}

% Note
% lattice label  -- defined in Lattice class
% security label -- defined in SecType
% explain why complex security type is necessary
% show SecType and explain every constructor
% show MultiLattice explain each operator
% modular way of expanding

\begin{figure}[t]
\[
\st\ ::=\ \ell\ |\ \st\ {\mathbf{ref}}^{\ell}\ |\ (\st,\st)\ |\ ({\mathbf{either}}\ \st\ \st\ )^{\ell}\ |\ {\mathbf{high}}
\]
\caption{Security types}
\label{fig:cst}
\end{figure}

\begin{figure}[t]
\[
\typ\ ::= int\ |\ \typ\ ref\ |\ (\typ,\typ)\ |\ either\ \typ\ \typ 
\]
\caption{Types}
\label{fig:types}
\end{figure}

\begin{figure}[t]
\[
 \begin{array}{c}
   %\inference[]{\ell_1 \sqsubseteq \ell_2}{\ell_1 \leq \ell_2} \quad
   \inference[]{\sts{1}=\sts{2} \quad \ell_1 \sqsubseteq \ell_2}
               {\sts{1}\ \mathbf{ref}^{\ell_1} \sqsubseteq \sts{2}\ \mathbf{ref}^{\ell_2}} \quad
   \inference[]{\sts{1} \sqsubseteq \sts{3} \quad \sts{2} \sqsubseteq \sts{4} }
               {(\sts{1},\sts{2}) \sqsubseteq (\sts{3},\sts{4})} \\ \\
   \inference[]{\ell_1 \sqsubseteq \ell_2 \quad \sts{1} \sqsubseteq \sts{3} \quad \sts{2} \sqsubseteq \sts{4}}
               {(\mathbf{either}\ \sts{1}\ \sts{2})^{\ell_1} \sqsubseteq
                (\mathbf{either}\ \sts{3}\ \sts{4})^{\ell_2} }
 \end{array} 
\]
\caption{Subtyping}
\label{fig:subtyping}
\end{figure}

\begin{figure}[t]
\[
  \begin{array}{c}
  \inference[]{}{e(\ell)~->~\ell} \quad
  \inference[]{}{e(\st~\mathbf{ref}^\ell)~->~\ell} \quad
  \inference[]{e(\sts{1})~->~\ell_1 \quad e(\sts{2})~->~\ell_2}
              {e((\sts{1},\sts{2}))~->~\ell_1\sqcup \ell_2} \\ \\
  \inference[]{}
              {e((\mathbf{either}~\sts{1}~\sts{2})^\ell)~->~\ell} \quad
  \inference[]{}{e(\mathbf{high})~->~\top} \\
  \end{array}
\]
\caption{Function $e$}
\label{fig:extract}
\end{figure}

\begin{figure}[t]
\[
  \begin{array}{c}
  \inference[]{\ell_1\sqsubseteq \ell_2}{\tagup~\ell_1~\ell_2~->~\ell_2} \quad
  \inference[]{\ell_2\sqsubset \ell_1}{\tagup~\ell_1~\ell_2~->~\ell_1} \\ \\
  \inference[]{\ell_1\sqsubseteq \ell_2}{\tagup~\st~\mathbf{ref}^{\ell_1}~\ell_2~->~\st~\mathbf{ref}^{\ell_2}} \quad
  \inference[]{\ell_2\sqsubset \ell_1}{\tagup~\st~\mathbf{ref}^{\ell_1}~\ell_2~->~\st~\mathbf{ref}^{\ell_1}} \\ \\
  \inference[]{\tagup~\sts{1}~\ell~->~\sts{3} \quad \tagup~\sts{2}~\ell~->~\sts{4}}
              {\tagup~(\sts{1},\sts{2})~\ell~->~(\sts{3},\sts{4})} \quad
  \inference[]{\ell_1 \sqsubseteq \ell_2 \quad \tagup~\sts{1}~\ell_2~->~\sts{3} \quad \tagup~\sts{2}~\ell_2~->~\sts{4}}
              {\tagup~(\mathbf{either}~\sts{1}~\sts{2})^{\ell_1}~\ell_2~->~
                      (\mathbf{either}~\sts{3}~\sts{4})^{\ell_2}} \\ \\
  \inference[]{\ell_2 \sqsubset \ell_1 \quad \tagup~\sts{1}~\ell_2~->~\sts{3} \quad \tagup~\sts{2}~\ell_2~->~\sts{4}}
              {\tagup~(\mathbf{either}~\sts{1}~\sts{2})^{\ell_1}~\ell_2~->~
                      (\mathbf{either}~\sts{3}~\sts{4})^{\ell_1}} \quad
  \inference[]{}
              {\tagup~\mathbf{high}~\ell~->~\mathbf{high}}
  \end{array}
\]
\caption{Function $\tagup$}
\label{fig:tagup}
\end{figure}

\begin{figure}[t]
\[
   \begin{array}{c}
   \inference[]{\ell_1 \sqsubseteq \ell_2}{\decl~\ell_1~\ell_2~\ell_2~->~\ell_1} \quad
   \inference[]{\ell_2 \sqsubset \ell_1}{\decl~\ell_1~\ell_2~\ell_2~->~\ell_2} \\ \\
   \inference[]{\decl~\ell_1~\ell_2~\ell_2~->~\ell_3 \quad \decl~\sts{1}~\sts{2}~\ell_2~->~\sts{3}}
               {\decl~(\sts{1}~\mathbf{ref}^{\ell_1})~(\sts{2}~\mathbf{ref}^{\ell_2})~\ell_2~->~
                \sts{3}~\mathbf{ref}^{\ell_3}} \quad
   \inference[]{\decl~\sts{1}~\sts{3}~\ell~->~\sts{5} \quad \decl~\sts{2}~\sts{4}~\ell~->~\sts{6}}
               {\decl~(\sts{1},\sts{2})~(\sts{3},\sts{4})~\ell~->~(\sts{5},\sts{6})} \\ \\
   \inference[]{\decl~\ell_1~\ell_2~\ell_2~->~ \ell_3 \quad \decl~\sts{1}~\sts{3}~\ell_2~->~\sts{5} \quad
                \decl~\sts{2}~\sts{4}~\ell_2~->~\sts{6}}
               {\decl~(\mathbf{either}~\sts{1}~\sts{2})^{\ell_1}~(\mathbf{either}~\sts{3}~\sts{4})^{\ell_2}
                ~\ell_2~->~\mathbf{either}~\sts{5}~\sts{6})^{\ell_3}} \quad
   \inference[]{} 
               {\decl~\mathbf{high}~\st~\ell~->~\st} \\
   \end{array}
\]
\caption{Function $\decl$}
\label{fig:decl}
\end{figure}

A single security label in FlowHaskell is not sufficient to capture the security types required
by a reference whose identity and content may have different security levels.
Complex security types are introduced in FlowHaskellRef to resolve the limitation.
The complex security type, as in Figure~\ref{fig:cst}, is based on a lattice
, $l$, and provide different security labels for different data.
Each complex security label represents the security label of a data type, as
showed in Figure~\ref{fig:types}.
Data types that are not specified in Figure~\ref{fig:types} are protected by the security 
label $\ell$, which is a lattice label.
Abstract data type \co{SecType} parameterises over a 
lattice type and implements the complex security type in Figure~\ref{fig:cst}.
The definition of \co{SecType} is as below(in Lattice.hs):
\begin{verbatim}
data SecType l = SecLabel l | Ref (SecType l) l | 
                 Pair (SecType l) (SecType l) |
                 SecEither (SecType l) (SecType l) l |
                 SecHigh  
     deriving (Eq, Show)
\end{verbatim}
Type variable \co{l} is a lattice label defined by programmers.
Constructor \co{SecLabel} takes a lattice label and represents a security
label for data types \co{Int}, \co{Bool}, and all other types not
specified in Figure~\ref{fig:types}. 
Constructor \co{Ref} takes a complex security label for its content and
a lattice label for a reference's identity. The complex security label
for reference is based on the work of Pottier and Simonet~\cite{Pottier:Simonet:POPL02}.
Constructor \co{Pair} takes security labels for the first and the 
second element respectively. 
Constructor \co{SecEither} takes two security labels for the two constructors
of \co{Either} type in Haskell, \co{Left} and \co{Right}. The third parameter
is a lattice label protecting the identity of a \co{Either} type. It represents
the security level required to distinguish the two constructors.
Constructor \co{SecHigh}, corresponding to the $\mathbf{high}$, represents any 
complex security label with each lattice
label annotation being a top element in the lattice. The purpose of introducing
\co{SecHigh} is explained in Sec.~\ref{chap3:pure}.

Security type $\st$ exhibits a partial ordering among security labels of identical
constructors. The partial ordering of the underlying lattice can
be lifted to $\st$ and form several new lattices. For example, $(\ell_1,\ell_2)$ and $(\ell_3,\ell_4)$
are in a lattice because they are both a pair of lattice labels.
The ordering of new lattices are defined by a sub-typing relation as depicted in
Figure~\ref{fig:subtyping}.
Type class \co{MultiLattice} captures this idea and provides
required operations(in Lattice.hs).
\begin{verbatim}
class (Lattice l, Eq (a l), Show (a l)) 
        => MultiLattice a l where
  mlabel_top :: a l -> a l
  mlabel_bottom :: a l -> a l
  mlabel_label :: l -> a l -> a l
  mlabel_join :: a l -> a l -> a l
  mlabel_meet :: a l -> a l -> a l
  mlabel_leq :: a l -> a l -> Bool
  mextract :: a l -> l
  mext_join :: a l -> l
  mext_meet :: a l -> l
  mlabel_tag :: a l -> l -> a l
  mlabel_decl :: a l -> a l -> l -> a l
\end{verbatim} 
Type variable \co{a} is an abstract data type that represents a multiple lattice
structure based on a lattice \co{l}. In the rest of the section, we assume type
variable \co{a} is \co{SecType} and explained each method in this context.
Class method \co{mlabel\_top} and \co{mlabel\_bottom} take a complex security label
and produce top and bottom elements in the same lattice respectively.
They are special cases of method \co{mlabel\_label} which produces an element in the
same lattice as the second complex security label but with all security label annotations being $l$.
Method \co{mlabel\_join} and \co{mlabel\_meet} return the join and the meet of two
elements in the same lattice respectively.
Method \co{mlabel\_leq} implements the sub-typing relation showed in 
Figure~\ref{fig:subtyping}. This sub-typing relation corresponds to the partial
ordering relation in lattice for each constructor. For security label $\ell$ of $\st$,
the partial ordering relation is the same as the underlying lattice, $l$.
Method \co{mextract} implements operator $e$ in Figure~\ref{fig:extract}. It returns 
a lattice label that represents the security level of a information flow and is used
to eliminate {\em implicit} flow.
Class method \co{mext\_join}(\co{mext\_meet}, resp.) returns a lattice label which is
the join(the meet, resp.) of label annotations inside of a complex security label. 
Method \co{mlabel\_tag} implements operator $\tagup$ in Figure~\ref{fig:tagup} and
method \co{mlabel\_decl} implements operator $\decl$ in Figure~\ref{fig:decl}. The 
second complex security label of \co{mlabel\_decl} has identical constructors to the 
first complex security label except when the first one is $\mathbf{high}$. Each label
annotation in the second complex security label is the same as the third lattice 
label.
%\co{SecType} is an instance of \co{MultiLattice}. The following shows implementation
%of constructor \co{Ref} as example(in Lattice.hs): 
%\begin{verbatim}
%instance Lattice l => MultiLattice SecType l where
%  mlabel_label l (Ref t _) = Ref (mlabel_label l t) l
%  mlabel_leq (Ref st1 lr1) (Ref st2 lr2) = 
%    label_leq lr1 lr2 && mlabel_leq st1 st2
%  mlabel_leq SecHigh (Ref st2 lr2) = 
%    (label_leq label_top lr2) && (mlabel_leq SecHigh st2)
%  mlabel_leq (Ref st1 lr1) SecHigh = 
%    (label_leq lr1 label_top) && (mlabel_leq st1 SecHigh)
%  mextract (Ref d l) = (mextract d) `label_join` l
%  mextract (SecEither s t) = 
%    (mextract s) `label_join` (mextract t)
%  ...
%\end{verbatim}
%In method \co{mlabel\_label}, the resulting \co{Ref} applies \co{mlabel\_label} again
%to obtain security label for its content and takes label \co{l} as its own identity.
%To find order of two \co{Ref} security labels, labels of their identity are compared
%and so are security labels of their contents. Every structure of \co{SecType} is in
%partial order relation with constructor \co{SecHigh}. To compare with \co{SecHigh},
%any field of lattice label is compared with \co{label\_top}, and \co{SecHigh} 
%propagates to any field of multiple lattice label. Method \co{mextract} applying to
%\co{Ref} returns the security label of its identity. One more interesting case is
%when \co{mextract} applies to \co{SecEither}. Because \co{SecEither} is actually
%a single value and aimed for a more accurate description of \co{Either} type, there 
%is no single identity security label for \co{SecEither}. Flow security label of a
%\co{SecEither} is thus to join two flow security labels of possible values.

Programmers only need to define a lattice type to achieve their goals as in FlowHaskell. 
Complex security type \co{SecType} is part of the security language and 
maintained by library providers. Any new complex security label can be easily extended 
in \co{SecType} by adding a new type constructor and implementations of the methods
in \co{MultiLattice}. No change of code is required for old programs to 
take advantage of new complex security labels.

