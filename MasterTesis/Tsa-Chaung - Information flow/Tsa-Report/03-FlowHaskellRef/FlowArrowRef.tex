
\section{FlowArrowRef}

% Note
% explain how it extend from FlowArrow
% show definition and explain each field
% type system of FlowArrowRef

To deal with reference, some new types of fields or constraints are required to
guarantee no information leak due to reference manipulation. For example, a field
to keep track of side effect is now necessary because reference creation and 
modification produce side effect. The constraint field of \co{FlowArrow}, 
described in Sec.~\ref{chap3:flowarrow}, is not sufficient. One solution is to extend new
fields directly in \co{FlowArrow}, but it breaks integrity of \co{FlowArrow} and
a lot of related operators and help functions have to be redefined. 
As a new embedded language interface, \co{FlowArrowRef} extends \co{FlowArrow} in 
a modular way, in which \co{FlowArrow} becomes a field of \co{FlowArrowRef} and 
other necessary fields due to reference manipulation are added in \co{FlowArrowRef}. 
This makes old functions of \co{FlowArrow} can be reused and provides a better 
abstraction for related functions of \co{FlowArrowRef} to concentrate on handling
new constraints.

\subsection{Conceptual Abstract Data Type}
\label{chap3:flowarrowref:concept}
The definition of \co{FlowArrowRef} is as below:
\begin{verbatim}
data FlowArrowRef l a b c = 
     FARef { flowarrow :: FlowArrow l a b c
           , pc :: l
           , constraintsRef :: [Constraint (SecType l)]
           , gconstraintsRef :: [GConstraint SecType l]
           }

data Constraint l = LEQ l l | USERGEQ l | IS l l 

data GConstraint a l = GUARD l (a l) | GLEQ (a l) l
\end{verbatim}
\co{FlowArrowRef} has four type parameters. Type variable \co{l} is a lattice label,
variable \co{a} is an arrow which performs desired computation, and variables \co{b} 
and \co{c} are types
of input and output respectively. Field \co{flowarrow} is of type \co{FlowArrow} that
handles all flows and constraints irrelevant to reference manipulation. Field \co{pc} 
is a lattice label tracking lowest label of side effects in a computation. Field 
\co{constraintsRef} and \co{gconstraintsRef} are new constraints required. Although
field \co{constraintsRef} is the same type as field \co{constraints} in \co{FlowArrow},
they are separated for modularity. A new constructor of \co{Constraint},
\co{IS~l_1~l_2}, assures \co{l_1} has an identical structure to \co{l_2}. Data type 
\co{GConstraint} is a set of constraints that compare a lattice label and flow
security label of a \co{SecType}. It differs from \co{Constraint} which involves
two \co{SecType}s.

Notice this is not the final definition of \co{FlowArrowRef}. Because of the introduction
of complex security type, complex security labels are required to be passed and remembered 
precisely through computation. Mechanisms in original \co{FlowArrow} is insufficient that 
only a lattice label is used for security label of any data type. Therefore, unification 
is adopted to resolve the limitation. All unification related designs are explained later in 
Sec.~\ref{chap3:unification}.
Only conceptual framework of \co{FlowArrowRef} is illustrated here.

\subsection{Type System}
\label{chap3:flowarrowref:typesystem}


\begin{figure*}[t]
  \[\begin{array}{c}
    \inference[$\mathit{PURE}$]{f~:~\typn{1}~->~\typn{2}}
                               {\top,\emptyset,\emptyset\proves \cod{pure}~f~:~
                                \res{\typn{1}}{\sts{1}}~->~\res{\typn{2}}{\mathbf{high}}} \\ \\

    \inference[$\mathit{SEQ}$]{pc_1,\Delta_1,\Theta_1\proves f_1~:~\res{\typn{1}}{\sts{1}}~->~\res{\typn{x}}{\sts{2}}\quad 
                    pc_2,\Delta_2,\Theta_2\proves f_2~:~\res{\typn{x}}{\sts{3}}~->~\res{\typn{4}}{\sts{4}}}
                   {pc_1\sqcap pc_2,\Delta_1\cup \Delta_2,\Theta_1\cup \Theta_2\cup \{\sts{2}\sqsubseteq \sts{3}\}
                    \proves f_1~>\negthickspace>\negthickspace>~f_2~:~
                    \res{\typn{1}}{\sts{1}}~->~\res{\typn{4}}{\sts{4}}} \\ \\


    \inference[$\mathit{FIRST}$]{pc,\Delta,\Theta\proves f~:~(\typn{1},\sts{1})~->~(\typn{2},\sts{2})}
                     {pc,\Delta,\Theta\proves \cod{first}~f~:~
                      \res{(\typn{1},\typn{3})}{(\sts{1},\sts{3})}~->~
                      \res{(\typn{2},\typn{4})}{(\sts{2},\sts{4})}} \\ \\
    
    \inference[$\mathit{SECOND}$]{pc,\Delta,\Theta\proves f~:~(\typn{1},\sts{1})~->~(\typn{2},\sts{2})}
                      {pc,\Delta,\Theta\proves \cod{second}~f~:~
                       \res{(\typn{3},\typn{1})}{(\sts{3},\sts{1})}~->~
                       \res{(\typn{3},\typn{2})}{(\sts{3},\sts{2})}} \\ \\

    \inference[$\mathit{PAR1}$]{pc_1,\Delta_1,\Theta_1\proves f_1~:~
                         \res{\typn{1}}{\sts{1}}~->~\res{\typn{2}}{\sts{2}}\quad 
                         pc_2,\Delta_2,\Theta_2\proves f_2~:~
                         \res{\typn{3}}{\sts{3}}~->~\res{\typn{3}}{\sts{4}}}
                        {pc_1\sqcap pc_2,\Delta_1\cup \Delta_2,\Theta_1\cup \Theta_2
                         \proves f_1~*\negthickspace*\negthickspace*~f_2~:~
                         \res{(\typn{1},\typn{3})}{(\sts{1},\sts{3})}~->~
                         \res{(\typn{2},\typn{4})}{(\sts{2},\sts{4})}} \\ \\

    \inference[$\mathit{PAR2}$]{pc_1,\Delta_1,\Theta_1\proves f_1~:~
                        \res{\typn{x}}{\sts{1}}~->~\res{\typn{2}}{\sts{2}}\quad 
                        pc_2,\Delta_2,\Theta_2\proves f_2~:~
                        \res{\typn{x}}{\sts{3}}~->~\res{\typn{4}}{\sts{4}}}
                       {pc_1\sqcap pc_2,\Delta_1\cup \Delta_2,\Theta_1\cup \Theta_2
                        \proves f_1~\arrowop{\&}~f_2~:~
                        \res{\typn{x}}{(\sts{1}\sqcap \sts{3})}~->~
                        \res{(\typn{2},\typn{4})}{(\sts{2},\sts{4})} } \\ \\
    \end{array}
   \]
\caption{Type system of methods in type class \co{Arrow}}
\label{fig:flowarrowref:typesystem0}
\end{figure*}

\begin{figure*}[t]
\[ \begin{array}{c}
    \inference[$\mathit{LEFT}$]{pc,\Delta,\Theta\proves f~:~
                     \res{\typn{1}}{\sts{1}}~->~\res{\typn{2}}{\sts{2}}}
                     {pc,\Delta,\Theta\proves \cod{left}~f~:~
                     \res{either\ \typn{1}\ \typn{3}}{(\mathbf{either}\ \sts{1}\ \sts{3})^\ell}~->~
                     \res{either\ \typn{2}\ \typn{3}}{(\mathbf{either}\ \sts{2}\ \sts{3})^\ell}} \\ \\

    \inference[$\mathit{RIGHT}$]{pc,\Delta,\Theta\proves f~:~
                     \res{\typn{1}}{\sts{1}}~->~\res{\typn{2}}{\sts{2}}}
                     {pc,\Delta,\Theta\proves \cod{right}~f~:~
                     \res{either\ \typn{3}\ \typn{1}}{(\mathbf{either}\ \sts{3}\ \sts{1})^\ell}~->~
                     \res{either\ \typn{3}\ \typn{2}}{(\mathbf{either}\ \sts{3}\ \sts{2})^\ell}} \\ \\

    \inference[$\mathit{CHOICE1}$]{pc_1,\Delta_1,\Theta_1\proves f_1~:~
                       \res{\typn{1}}{\sts{1}}~->~\res{\typn{2}}{\sts{2}} \quad 
                       pc_2,\Delta_2,\Theta_2\proves f_2~:~
                       \res{\typn{3}}{\sts{3}}~->~\res{\typn{4}}{\sts{4}}}
                       {pc_1\sqcap pc_2,\Delta_1\cup \Delta_2\cup constraint1
                       , \Theta_1\cup \Theta_2
                       \proves  f_1~\arrowop{+}~f_2~:~flow1 }\\ \\

    flow1~=~\res{either\ \typn{1}\ \typn{3}}{(\mathbf{either}\ \sts{1}\ \sts{3})^\ell}~->~
            \res{either\ \typn{2}\ \typn{4}}{(\mathbf{either}\ \sts{2}\ \sts{4})^\ell} \\ \\

    constraint1 = \{(\mathbf{either} \sts{1}\ \sts{3})^\ell\sleql (pc_1\sqcap pc_2),
                   (\mathbf{either}\ \sts{1}\ \sts{3})^\ell\sleql \ell \} \\ \\

    \inference[$\mathit{CHOICE2}$]{pc_1,\Delta_1,\Theta_1\proves f_1~:~
                         \res{\typn{1}}{\sts{1}}~->~\res{\typn{x}}{\sts{2}} \quad 
                         pc_2,\Delta_2,\Theta_2\proves f_2~:~
                         \res{\typn{3}}{\sts{4}}~->~\res{\typn{x}}{\sts{4}}} 
                         {pc_1\sqcap pc_2,\Delta_1\cup \Delta_2\cup constraint2
                         , \Theta_1\cup \Theta_2
                         \proves f_1~|||~f_2~:~flow2} \\ \\

    flow2~=~\res{either\ \typn{1}\ \typn{3}}{(\mathbf{either}\ \sts{1}\ \sts{3})^\ell}~->~
            \res{\typn{x}}{(\sts{2}\sqcup \sts{4})} \\ \\

    constraint2 = \{(\mathbf{either} \sts{1}\ \sts{3})^\ell\sleql (pc_1\sqcap pc_2),
                   (\mathbf{either}\ \sts{1}\ \sts{3})^\ell\sleql e(\sts{2}\sqcup \sts{4}),  
                   \ell\guard \sts{1}, \ell\guard \sts{3} \} \\ \\
%    \inference[$\mathit{LOOP}$]{pc,\Delta,\Theta\proves f~:~(\tau_1^s,\tau_2^s)^l~->~(\tau_3^s,\tau_4^s)^{l'}}
%                    {pc,\Delta\cup \{l\lhd\ \tau_1^s, l'\lhd\ \tau_3^s\},
%                     \Theta\cup \{\tau_4^s\sqsubseteq \tau_2^s\}\proves \cod{loop}\ f~:~\tau_1^s~->~\tau_3^s} \\ \\
%
    \end{array}
  \]
\caption{Type system of methods in type class \co{ArrowChoice}}
\label{fig:flowarrowref:typesystem1}
\end{figure*}

\begin{figure}[t]
\[
  \begin{array}{c}
    \inference[$\mathit{TAG}$]{}
                   {\top,\emptyset,\emptyset \proves \cod{tagRef}\ \ell~:~
                    \res{\typ}{\st}~->~\res{\typ}{\tagup (\st,\ell)}} \\ \\

    \inference[$\mathit{DECL}$]{}
                    {\top,\emptyset,\{\st\sqsubseteq \cod{user}\}\proves 
                     \cod{declassifyRef}~\ell~:~
                     \res{\typ}{\st}~->~\res{\typ}{\decl (\st,\rho(\typ,\ell),\ell)}}  \\ \\

    \inference[$\mathit{LOWER}$]{\top,\Delta,\Theta\proves f~:~
                    \res{\typn{1}}{\sts{1}}~->~\res{\typn{2}}{\sts{2}}}
                    {\top,\{\ell\lleqs \sts{1}\} ,\emptyset\proves \cod{lowerA}~\ell~f~:~
                    \res{\typn{1}}{\sts{1}}~->~\res{\typn{2}}{\rho(\typn{2},\ell)}} \\
  \end{array}
\]
\caption{Type system of new arrow combinators}
\label{fig:flowarrowref:typesystem2}
\end{figure}

\begin{figure*}[t]
 \[
   \begin{array}{c}
   \inference[Certify]{pc,\Delta,\Theta\proves f~:~\res{\typn{1}}{\sts{1}}~->~\res{\typn{2}}{\sts{2}} \quad 
                       \sts{in}\sleql pc\quad
                       \sts{in}\sqsubseteq \sts{1} \\ \sts{2}\sqsubseteq \sts{out} \quad
                       L\proves \Delta \quad L\proves \Theta[\ell_u/\cod{user}]}
                      {L,\ell_u\proves \cod{certifyRef}~f~:~\res{\typn{1}}{\sts{in}}~->~\res{\typn{2}}{\sts{out}}}  \\ 
   \end{array}
 \]
\caption{Type system of certify}\label{fig:flowarrowref:certify}
\end{figure*}

\begin{figure}[t]
\[
  \begin{array}{c}
  \inference[]{\ell'\sqsubseteq \ell}{\ell'\sleql \ell} \quad
  \inference[]{\ell'\sqsubseteq \ell\quad \st\sleql \ell}{\st\ \mathbf{ref}^{\ell'}\sleql \ell} \quad
  \inference[]{\sts{1}\sleql \ell \quad \sts{2}\sleql \ell}{(\sts{1},\sts{2})\sleql \ell} \\ \\
  \inference[]{\ell'\sqsubseteq \ell\quad \sts{1}\sleql \ell \quad \sts{2}\sleql \ell}
              {(\mathbf{either}~\sts{1}~\sts{2})^{\ell'}\sleql \ell} \quad
  \inference[]{\top\sqsubseteq \ell}{\mathbf{high}\sleql \ell} \\
  \end{array}
\]
\caption{Constraint $\sleql$}
\label{fig:flowarrowref:sleql}
\end{figure}

\begin{figure}[t]
\[
  \begin{array}{c}
  \inference[]{\ell\sqsubseteq \ell'}{\ell\lleqs \ell'} \quad
  \inference[]{\ell\sqsubseteq \ell' \quad \ell\lleqs \st}{\ell \lleqs \st\ \mathbf{ref}^{\ell'}} \quad
  \inference[]{\ell\lleqs \sts{1} \quad \ell\lleqs \sts{2}}{\ell\lleqs (\sts{1},\sts{2})} \\ \\
  \inference[]{\ell\sqsubseteq \ell' \quad \ell\lleqs \sts{1} \quad \ell\lleqs \sts{2}}
              {\ell\lleqs (\mathbf{either}~\sts{1}~\sts{2})^{\ell'}} \quad
  \inference[]{\ell\sqsubseteq \top}{\ell\lleqs \mathbf{high}}
  \end{array}
\]
\caption{Constraint $\lleqs$}
\label{fig:flowarrowref:lleqs}
\end{figure}

\begin{figure}[t]
\[
   \inference[]{\ell\sqsubseteq e(\st)}{\ell\guard \st}
\]
\caption{Constraint $\guard$}
\label{fig:flowarrowref:guard}
\end{figure}

\begin{figure}[t]
 \[
  \begin{array}{c}
  \inference[]{}{\rho(int,\ell)~->~\ell} \quad
  \inference[]{\rho(\typ,\ell)~->~\sts{1}}
              {\rho(\typ~ref,\ell)~->~\sts{1}~\mathbf{ref}^{\ell}} \\ \\
  \inference[]{\rho(\typn{1},\ell)~->~\sts{1} \quad \rho(\typn{2},\ell)~->~\sts{2}}
              {\rho((\typn{1},\typn{2}),\ell)~->~(\sts{1},\sts{2})} \quad
  \inference[]{\rho(\typn{1},\ell)~->~\sts{1} \quad \rho(\typn{2},\ell)~->~\sts{2}}
              {\rho(either~\typn{1}~\typn{2},\ell)~->~(\mathbf{either}~\sts{1}~\sts{2})^{\ell}} \\
  \end{array}
 \]
\caption{Function $\rho$}
\label{fig:deduce}
\end{figure}

The type system conceptually implemented by \co{FlowArrowRef} is depicted in 
Figure~\ref{fig:flowarrowref:typesystem0}, Figure~\ref{fig:flowarrowref:typesystem1},
and Figure~\ref{fig:flowarrowref:typesystem2}.
The type judgement has following form:
\[
pc,\Delta,\Theta\proves f~:~\res{\typn{1}}{\sts{1}}~->~\res{\typn{2}}{\sts{2}}
\]
where $pc$ is lower bound of side effects in computation \co{f}, 
$\Delta$ denotes guard constraints as \co{GConstraint} in \co{FlowArrowRef}.
Environment $\Theta$ represents constraints of \co{Constraint} that appear both in
\co{FlowArrow} and \co{FlowArrowRef}. The input and output types of \co{f} are
$\typn{1}$ and $\typn{2}$, while the input and output security labels
are $\sts{1}$ and $\sts{2}$, respectively.

Combinator \co{pure} lifts a Haskell function into \co{FlowArrowRef}. It takes any
input security label $\sts{1}$ and returns $\mathbf{high}$
as the output security label. 
No information leaks in \co{pure} because everything
becomes top secret afterward. The reason of this design choice are explained 
in Sec.~\ref{chap3:pure}. Rule (\co{SEQ}) ensures output security label
of the first computation is less than or equal to input security label of the second computation via
constraint $\sqsubseteq$. The new $pc$ is the meet of the $pc$s of two underlying computations 
and records the lower bound of side effects in both computations. The new constraints are the
union of the original constraints in both computations. The definition of $pc$ and constraints 
are the same for rules involving two computations.
In rule (\co{FIRST}), (\co{SECOND}), and (\co{PAR1}), both input and output types and security labels
are wrapped in a pair constructor.
In rule (\co{Par2}), both underlying computations should have the same input type. The final
input security label becomes the meet of two input security labels.

In Figure~\ref{fig:flowarrowref:typesystem1}, rule (\co{LEFT}) and (\co{RIGHT}) wrap input and output
types and security labels in either constructors. The security label for the identity of an either
type keeps the same from input to output.
In rule (\co{CHOICE1}), two constraints are generated. Constraint $\sleql$ is defined in 
Figure~\ref{fig:flowarrowref:sleql}.
The first constraint $(\mathbf{either} \sts{1}\ \sts{3})^\ell\sleql (pc_1\sqcap pc_2)$
says the security label of the lowest side effect in two underlying computations should be 
above or equal to any security label annotations of inputs. This prevents the case illustarted
in the following pseudocode:
\begin{verbatim}
 pure (\x -> if x>0 then (Left x) else (Right x)) >>>
 ((l := 1) +++ (l := 0))
\end{verbatim}
Assignment to $l$ is like writing reference $l$, which is explained in 
Sec.~\ref{chap4:reference}. Assume $l$ is a low reference, the side effect of the writes are also
low. However, the input of (\arrowop{+}) from the result of \co{pure} is $\top$ and information is 
leaked in the content of reference $l$.
The second constraint $(\mathbf{either}\ \sts{1}\ \sts{3})^\ell\sleql \ell$ demands the output
security label is above or equal to any input security label annotations. The identity label of 
$\mathbf{either}$, $\ell$, is sufficient because the security labels of the contents should be 
at least the same or above the identity security label when extracted from the constructors.
A counter example that is simple and conveys the same idea is given in the explaination of 
rule (\co{CHOICE2}).

In rule (\co{CHOICE2}), there four constraints created. The first is the smae and has already been 
explained in rule (\co{CHOICE1}). The second constraint 
$(\mathbf{either}\ \sts{1}\ \sts{3})^\ell\sleql e(\sts{2}\sqcup \sts{4})$
requires the security label representing the security context of outputs is above or equal to
all security label annotations of the inputs. Operator $e$ is defined in Figure~\ref{fig:extract}.
The following program illustrates the necessity of this constraint.
\begin{verbatim}
 pure (\_ -> 0) >>>
 pure (\x -> if x > 10 then (Left x) else (Right x)) >>>
      ((lowerA LOW (pure(\_ -> 1))) 
       ||| 
       (lowerA LOW (pure (\_-> 0))))
\end{verbatim}
Assume we have two labels in the lattice, \co{LOW} and \co{HIGH}.
Combinator \co{lowerA} in the program above produces output security labels with each label 
annotation being \co{LOW}(see Sec.~\ref{chap3:lower} for detail information of \co{lowerA}).
The output of \co{pure}, nevertheless, is of security type $\mathbf{high}$.
The attackers thus have information of \co{HIGH} variable \co{x} by observing \co{LOW} outputs.
The remaining two constraints $\ell\guard \sts{1}$ and $\ell\guard \sts{3}$ are required because
combinator ($|||$) takes the contents out of either constructors to perform computations. 
They guarantees the security label of the contents is equal to or above the security level of
the program context, which is $\ell$.
The constraint $\guard$ is defined in Figure~\ref{fig:flowarrowref:guard}. 

Figure~\ref{fig:flowarrowref:typesystem2} shows the type system of new combinators that are not
in standard {\em arrow} interface. Combinator \co{tagRef} takes a lattice label, $\ell$, and produces an output
with security label which has the same constructors as the input security label but with each label annotation 
higher or equal to $\ell$. This is implemented by function $\tagup$, as defined in Figure~\ref{fig:tagup}.
Combinator \co{declassifyRef} takes a lattice label, $\ell$, and declassifies the input security label. 
Each label annotation in the input security label which is higher than $\ell$ is downgraded to $\ell$.
This is done by function $\decl$ in Figure~\ref{fig:decl}. A constraint $\st\sqsubseteq \cod{user}$
quarantees the authority of the code, denoted by \co{user}, is equal or higher than the input
security label to perform the declassification. 
Combinator \co{lowerA} takes a lattice label
and a \co{FlowArrowRef}, and transform the output security label of the computation based on the output
type. Operator $\rho$, defined in Figure~\ref{fig:deduce}, generates a security label with every label 
annotation being $\ell$. A constraint $\ell\lleqs \sts{1}$ requires all the label annotations in the 
input security label are higher or equal to the given security label, $\ell$. Constraint $\lleqs$ is
defined in Figure~\ref{fig:flowarrowref:lleqs}.
The detail information of combinator \co{lowerA} is explained in Sec.~\ref{chap3:lower}.

The type system for \co{certifyRef} is described in Figure~\ref{fig:flowarrowref:certify}. Type judgement
$L,\ell_u\proves f~:~\res{\typn{1}}{\sts{in}}~->~\res{\typn{2}}{\sts{out}}$ says given a lattice $L$ and 
an authority privilege label $\ell_u$, computation $f$ takes an input of type $\typn{1}$ with security 
label $\sts{in}$, and returns an output of type $\typn{2}$ with security label $\sts{out}$.
Computation $f$ is well-typed if all the constraints in the premise are satisfied.

