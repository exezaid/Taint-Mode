\documentclass[oribibl]{llncs}
\usepackage{times}
\usepackage{stmaryrd,latexsym,amsmath,amssymb}
%\usepackage{proof}
\usepackage{wrapfig}
\usepackage{url}
\usepackage{semantic} % \inference
\usepackage{mathpartir} % display of inference rules
\usepackage{enumerate}
\usepackage{listings}
  

% program text
\newcommand{\cod}[1]{\mathtt{#1}}


% security type with index
\newcommand{\sts}[1]{s_{#1}^{\mathbb{L}}}
% security type without index
\newcommand{\st}{s^{\mathbb{L}}}


% symbol of constraint IS
\newcommand{\is}{\sim}
% symbol of gconstraint
\newcommand{\guard}{\lhd}
\newcommand{\sleql}{\LHD}
\newcommand{\lleqs}{\preceq}
% tag constraint
\newcommand{\tagup}{\uparrow}
% declassification constraint
\newcommand{\decl}{\downarrow}
% lifting function
\newcommand{\lift}{\nearrow}
% basic type without index
\newcommand{\typ}{\tau}
% basic type with index
\newcommand{\typn}[1]{\tau_{#1}}
% result of type system
\newcommand{\res}[2]{{#1}\mid {#2}}
% prove symbol
\newcommand{\proves}{\ensuremath{\vdash}}
% arrow combinator
\newcommand{\arrowop}[1]{$#1\negthickspace #1\negthickspace #1$}


% security label
\newcommand{\slabel}[1]{$\cod{#1}$}

% if-then-else
\newcommand{\ifthenelse}[3]{\cod{if}~#1~\cod{then}~#2~\cod{else}~#3}

% config of ArrowRef commands
\newcommand{\llocal}{\langle\hspace{-2.4 pt}|}
\newcommand{\rlocal}{|\hspace{-2.4 pt}\rangle}
\newcommand{\config}[2]{\llocal #1,#2 \rlocal}

\pagestyle{plain}

\title{A Taint Mode for Python via a Library}

\newcommand{\nametklass}{taint-aware }
\newcommand{\nameTklass}{Taint-aware }
\newcommand{\namefunc}{taint-aware }
\institute{
Universidad Tecnol\'{o}gica Nacional, Facultad Regional Santa Fe, Argentina \and 
Chalmers University of Technology, Sweden
}

\author{Juan Jos\'{e} Conti\inst{1} \and  Alejandro Russo\inst{2}}


%\author{[Anonymized for AppSec 2010]}
%\institute{\vspace{-5ex}}

\newif\ifexpandproofs
\expandproofsfalse


\begin{document}
\lstset{
        %language=Python, 
        stringstyle=\ttfamily, 
        showstringspaces=false,
        extendedchars=true,  
}

%\lstdefinelanguage{Python}
%{morekeywords={cat,python,rm}}


\maketitle
\thispagestyle{plain}




\begin{abstract}
Vulnerabilities in web applications present threats to on-line systems.
SQL injection and cross-site scripting attacks are among the
most common threats found nowadays. These attacks are
often result of improper or none input validation. 
To help discover such vulnerabilities, 
popular web scripting 
languages like Perl, Ruby, PHP, 
and Python perform taint analysis.
Such analysis is often 
implemented as an execution monitor, where the interpreter
needs to be adapted to provide a taint mode. 
% The analysis is usually implemented using an static (e.g. type
% systems) or dynamic (e.g. execution monitors) techniques. 
% In the latter case, the Perl, Ruby, PHP, and Python interpreters have
% been adapted to provide a taint mode. 
However, modifying interpreters might be a major task in its own
right. In fact, it is very probably that  
new releases of interpreters require to 
be adapted to provide a taint mode.
% where the analysis is performed. 
Differently from previous approaches, 
we show how to provide taint analysis for Python via a library
 written entirely in Python, and thus avoiding modifications in the interpreter.
The concepts of classes, decorators and dynamic dispatch
makes our solution lightweight, easy to use, and particularly neat.
With minimal or none effort, the library can be adapted
to work with different Python interpreters.
\end{abstract}
 
% 2 Page
\section{Introduction}
\label{sec:intro}

%% Situation: web applications widely use and no security still
Over the past years, there has been a significant increase 
%on the 
%tasks or activities performed on-line. 
on the number of activities performed on-line.
Users can do almost everything  
using a web browser (e.g. 
watching videos, listening to music, banking, booking flights, planing 
trips, etc). Considering the size of Internet and its number of users, 
%the amount of 
%people that use it, 
web applications are probably among the most used pieces of
software nowadays.
%in our modern society. 
Despite its wide use, web applications suffer 
from vulnerabilities that permit attackers to 
steal confidential data, break integrity of systems, 
and affect availability of services. 
When development of web applications is done 
with little or no
security in mind,  
the presence of security holes increases dramatically.
Web-based vulnerabilities have already  
outplaced those of all other platforms 
\cite{StateWebSecurity} 
and there are 
no reasons to think that this tendency has changed
\cite{FFAWebSecurity}.

%% Top attacks 
According to OWASP \cite{OWASP:Top10:2007}, 
cross-site scripting (XSS) and SQL injection (SQLI) attacks are among 
the most common vulnerabilities on web applications.
Although these attacks are classified differently, they are produced 
by the same 
reason: \emph{user supplied data is sent to sensitive sinks 
without a proper sanitation}.
For example, 
when a SQL query is constructed using an unsanitize string provided by a user, 
SQL injection attacks are likely to occur.
%, where the sensitive
%sink is the database engine. 
To harden applications against these attacks, 
the implementations of some popular web scripting languages 
%(Perl, Ruby, PHP,  and Python) 
perform taint analysis in a form 
of execution monitors \cite{Perl,Ruby}. In that manner, not only run 
interpreters code, but they also perform security checks. Taint
analysis can also be provided through static analysis \cite{WebSSARI,Jovanovic06pixy:a}. 
Nevertheless, execution monitors usually produce
less false alarms 
than traditional static techniques
\cite{Sabelfeld:Russo:PSI09}. 
%In other words, 
%execution monitors are likely to be more precise than traditional static techniques. 
In particular, static techniques cannot  
deal with dynamic code evaluation
%a common feature present in web
%scripting languages,  
without being too conservative. Most of the modern web scripting 
languages are capable to dynamically execute code. 
In this paper, we focus on dynamic techniques.




% Taint analsys
%A 
Taint analysis is an automatic approach to find vulnerabilities.
Intuitively, taint analysis restricts how tainted or untrustworthy 
data flow inside programs. Specifically, it constrains data 
to be untainted (trustworthy) 
or previously sanitized when 
reaching sensitive sinks. 
%In a web scenario, 
%user input is frequently considered as tainted, while 
%HTTP-request, SQL queries, and other security critical operations 
%are classified as sensitive sinks. 
Perl was the first scripting 
language to provide taint analysis 
as an special mode of the  
interpreter called \emph{taint mode} \cite{BekmanCholet2003}. 
Similar to Perl, some interpreters for 
 Ruby \cite{thomas2004prub}, PHP \cite{Nguyen05}, and 
recently Python \cite{KozlovPetukhov07} have been 
carefully modified to provide taint modes.
Adapting interpreters to incorporate taint analysis 
present two major drawbacks that directly 
impact on the adoption of this technology. 
Firstly, incorporating taint analysis into an interpreter 
might be a major task in its own right. Secondly, it is 
very probably that it is necessary to repeatedly adapt an  
interpreter at every new version or release of it.


\begin{wrapfigure}{r}{6.5cm}
%\begin{figure}[t]
\vspace{-25pt}
{\small{
\lstinputlisting[language=Python,numbers=right, numberstyle=\tiny]{email.py}
\caption{\label{fig:example}Code for \texttt{email.py}}
}}
\vspace{-15pt}
%\end{figure}
\end{wrapfigure}
%%% Solution / Contributions
Rather than modifying interpreters, we present how to provide
a taint mode for Python via a library written entirely in Python. 
Python is spreading fast inside
web development \cite{WikiPython}. 
%For instance, 
%the popular Wiki engine 
%MoinMoin (e.g. used by Ubuntu forums)
%and Youtube are mostly written in Python.  Among companies, 
%we can mention Google, which uses 
%the language for some of its search-engine internals at
%Google Groups, Gmail, and Google Maps.
Besides its successful use, Python presents 
some programming languages abstractions that makes possible 
to provide a taint mode via a library. For example, 
Python decorators \cite{Lutz:1999:LP} are a non-invasive and simple 
manner to declare sources of tainted data, sensitive sinks, and 
sanitation functions. Python's 
object-oriented and dynamic typing mechanisms allows the 
execution of the taint analysis with almost no modifications in the
source code. 
%\marginpar{Check later what about decorating Bool} 


The library provides a general method to enhance Python's built-in
classes with tainted values. 
%To demostrate the flexibility of our approach, 
%the library enhances the built-in classes for
%strings, integers, and floats. 
In general, taint
analysis tends to only consider strings or characters 
\cite{Perl,Nguyen05,Haldar05dynamictaint,KozlovPetukhov07,Futo07,SeoLam2010}.
% when 
%implemented as part of interpreters. 
In contrast, our library 
can be easily adapted to consider different built-in
classes
% as well as users' specific ones, 
and thus providing a taint
analysis for a wider set of data types. 
By only considering tainted strings, the library provides 
a similar analysis than in \cite{KozlovPetukhov07},
but without modifying the Python interpreter.
To the best of our knowledge, a
library for taint analysis has not been considered before. 



% 1 Page
\subsection{A motivating example}
\label{sec:example} 
%
%
%
%
\begin{figure}[t]
{\small{
\lstinputlisting[numbers=left, numberstyle=\tiny,language=Python,stringstyle=\ttfamily]{uses.txt}
\caption{\label{fig:uses} Different invocations for \texttt{email.py}}
}}
\end{figure}


We present an example to motivate the use of 
taint analysis in order to discover and repair 
vulnerabilities. 
The example considers an scenario of a web application
where users can send their remotely stored files by 
email.
Figure \ref{fig:example} shows the simple module \texttt{email.py}
that is responsible to perform such task.
For simplicity,
% and didactic purposes, 
the code takes the user input from the 
command line (lines 4 and 5) rather than from 
the web server.
%In a more realistic scenario, lines 4--8 would be part
%of a bigger piece of code rather than an small module.
Figure \ref{fig:uses} shows some invocations to the module from the
shell prompt. Line 1 shows a request from 
Alice to send her own file $\texttt{reportJanuary.xls}$ to 
her email address $\texttt{alice@domain.se}$. In this case, 
Alice's input produces a behavior which matches the 
intention of the module. 
In contrast, lines 2 and 3 show how attackers can 
provide particular inputs to exploit unintended 
or unforeseen behaviors of \texttt{email.py}.
%the module that were not intended or foreseen.
%More specifically, 
Line 2 exploits the fact that \texttt{email.py} 
was written assuming that 
users only request their own files. 
Observe how  
\texttt{devil@evil.com} gets information regarding 
users accounts by receiving the file \texttt{/etc/passwd}. 
%in his/her mailbox. 
Line 3 goes an step further 
and injects the command \texttt{rm -rf /} after sending the email.
%Observe that line 2 triggers the execution of 
%the command \texttt{mail -s "Re: your file" devil@evil.com  <
%  /etc/passwd ; rm -r  /} . 
These attacks demonstrate how, 
what was intended to be a simple email client,  
can become a web-based file browser or a terminal. To avoid 
these vulnerabilities, applications need to rigorously check 
for malicious data provided by users or any other 
untrustworthy source. 
Taint analysis helps to detect when 
data is not sanitize before it is used on security critical 
operations. In Section \ref{sec:secure:email}, we show how to harden 
\texttt{email.py} in order to reject the vulnerabilities 
shown in Figure \ref{fig:example}.
%data before it is used for
%security critical operations. 
%In the example above, our 
%library detects that variable \texttt{cmd} is tainted since it is 
%form from user input. As a result, 
%the security critical call to \texttt{os.system} cannot be executed 
%without introducing some procedure for sanitization. 
%In order for the 
%analysis to run, the code needs to be slightly marked with the 
% API of our library.


  

 
The paper is organized as follows.
%in the following way:
%Section \ref{sec:example}
%presents a motivating example for taint analysis.  
Section \ref{sec:library} outlines the library API.
Section \ref{sec:implementation} 
describes the most important implementation details of our approach.
Section \ref{sec:related} covers related work.  
Section \ref{sec:conclusions} provides some 
concluding remarks. 






% 2 Pages
\section{A library for taint analysis}
\label{sec:library}

%\begin{figure}[t]
\begin{wrapfigure}{r}{4.5cm}
\vspace{-30pt}
{\small{
\lstinputlisting[language=Python]{implicit.py}
\caption{\label{fig:implicit}An implicit flow}
}}
\vspace{-20pt}
\end{wrapfigure}
%\end{figure}

%%% Intro


%% Idea of taint analysis
On most situations, taint analysis %\cite{Perl,Ruby} 
propagates 
taint information on assignments.  
Intuitively, when the right-hand side of an assignment uses a tainted value, 
the variable appearing on the left-hand side becomes tainted.
%For instance, given that the strings  
%\texttt{t} and \texttt{u} are respectively tainted and untainted, 
% variable $\texttt{z}$ becomes tainted if
%$\texttt{z = t+u}$.
Taint analysis can be seen as an information-flow tracking 
mechanism for integrity \cite{Sabelfeld:Myers:JSAC}. 
In fact, taint analysis is just a mechanism to 
track explicit flows, i.e. direct flows of information 
from one variable to another. 
Taint analysis tends to ignore 
implicit flows \cite{Denning:Denning:Certification}, i.e. 
flows through the control-flow constructs of the language. 
Figure \ref{fig:implicit} presents an implicit 
flow. Variables \texttt{t} and \texttt{u} are 
tainted and untainted, respectively.
%, respectively. 
Observe that variable \texttt{u} 
is untainted after the execution of the branch since 
an untainted value ($\texttt{'a'}$ or
$\texttt{''}$) is assigned to it. Yet, the value of the tainted variable 
$\texttt{t}$ is copied into the untainted variable 
\texttt{u} when \texttt{t == 'a'}. 
It is not difficult to imagine 
programs that circumvent the taint analysis by
copying the content of tainted strings 
into untainted ones  by using implicit flows\cite{Russo:IOS}.
%Besides However, 
%taint analyses are still useful under certain circumstances.


%, which circumvents the taint analysis.
%Nevertheless, taint analyses are still useful under certain attacker 
%models. 

In scenarios where attackers has full control over the code 
(e.g. when the code is potentially malicious), implicit flows present
an effective way to circumvent the taint
analysis. In this case, the attackers' goal  is to craft the code and input 
data in order to circumvent security mechanisms. There is a large body
of literature on the area of language-based security regarding 
how to track implicit flows \cite{Sabelfeld:Myers:JSAC}. 


There exists scenarios where 
the code is non-malicious, i.e. written without malice. 
Despite the good intentions and experience of programmers, 
the code might still contain vulnerabilities 
%(such 
as the ones
described in Section \ref{sec:example}. The attackers' goal  
consists on craft input data in order to exploit 
vulnerabilities and/or corrupt data. In this scenario,
taint analysis certainly helps to discover vulnerabilities. 
%If taint analysis ignores implicit flows, 
How dangerous are implicit flows in non-malicious code? We argue that they
are frequently harmless \cite{Russo:IOS}. The reason for that 
relies on that non-malicious programmers
need to write a more involved, and rather unnatural,
code in order to, for instance, copy tainted strings into untainted ones. 
In contrast, to produce explicit flows, programmers simply
need to forget a call to some sanitization function. 
For the rest of the paper, we consider 
scenarios where the analyzed code is non-malicious. 



\subsection{Using the library}
\label{sec:using}
\begin{figure}[t]
%\begin{tabular}{l@{\hspace{10pt}}|@{\hspace{10pt}}l}
{\small{
\begin{minipage}[t]{0.5\linewidth}
\lstinputlisting[language=Python,numbers=left,numberstyle=\tiny]{api.txt} 
\end{minipage}
%& 
\hspace{40pt}
\begin{minipage}[t]{0.5\linewidth}
\lstinputlisting[language=Python,numbers=left,firstnumber=13,numberstyle=\tiny]{api2.txt} 
\end{minipage}
%\end{tabular}
\caption{\label{fig:API}API for taint analysis}
}}
\end{figure}


%From a developer point of view, 
The library is essentially a series of functions 
to mark  
what are the sources of untrustworthy data, sensitive sinks, and
sanitation functions. 
%in a given application.
Figure \ref{fig:API} illustrates how the API works.
Symbol $\texttt{...}$ is a place holder for code that 
is not relevant to explain the purpose of the API.
We assume that $\texttt{v}$ is a variable, $\texttt{d}$ is an
string or integer, and \texttt{f} is a user-defined function. Symbol
$\texttt{T}$ represents a tag. 
By default, tags can take values  
\texttt{XSS}, \texttt{SQLI}, \texttt{OSI} (Operating System Injection), and
\texttt{II} (Interpreter Injection).
These values are used to indicate specific vulnerabilities that 
could be exploited by tainted data.
For instance, 
tainted data associated with tag \texttt{SQLI} is likely to exploit 
SQL injection vulnerabilities.
%%
%%
Function $\texttt{taint}$ is used to taint values. 
For example, line 1 taints variable \texttt{d}.
The call to \texttt{untrusted(web.input)} 
establishes that the results produced by \texttt{web.input} are tainted.
Line 5 shows how \texttt{untrusted} can be used  to mark 
the values returned by function \texttt{f}
as untrustworthy. Observe the use of the decorator syntax
 (\texttt{@untrusted}).
%indicate that it returns
%tainted values. 
Function  
\texttt{untrusted\_args} is used to indicate which functions' arguments must be
tainted. This primitive is particularly useful 
when programming frameworks require 
to redefine some methods in order to get information from  
external sources. 
%into the application.
% into the application. 
As an example, Twisted\cite{Twisted}, a framework to 
develop network applications, 
calls method \texttt{lineReceived} from the class
\texttt{LineOnlyReceiver} 
every time that an string is received from the network.
Lines 9--12 extend the class \texttt{LineOnlyReceiver}
and implement the method \texttt{lineReceived}. 
Line 10 taints the data that Twisted takes 
from the network. 
Functions \texttt{taint}, \texttt{untrusted}, and
\texttt{untrusted\_args} associate all the tags to the
tainted values. After all, 
untrustworthy data might exploit any kind of vulnerability. 
Line 13 marks \texttt{eval} as a sensitive sink. If eval receives 
a tainted data with the tag \texttt{T},
a possible vulnerability T is reported. Line 15 shows how to 
use \texttt{ssink} with the decorator syntax. 
Line 19 shows how \texttt{cleaner} establishes that 
function \texttt{wash} sanitizes data with tag \texttt{T}. 
As a result of that, function \texttt{w} removes tag \texttt{T}
from tainted values. Line 21 shows the use
of \texttt{cleaner} with the decorator syntax.
{Sensitive sinks and sanitization functions
can be associated with more than one kind of vulnerabilities 
by just nesting decorators,
i.e. \texttt{ssink(OSI)(ssink(II)(critical\_operation))}}.

\subsection{Hardening \texttt{email.py}}
\label{sec:secure:email}

\begin{wrapfigure}{r}{7.3cm}
\vspace{-30pt}
{\small{
%\begin{figure}[t]
\lstinputlisting[language=Python,numbers=right,numberstyle=\tiny]{safeclean.py}
\caption{\label{fig:example:secure} Secure version of module \texttt{email.py} }
\vspace{-10pt}
%\end{figure}
}}
\end{wrapfigure}

%To illustrate how our library works, 
We revise the example in Section \ref{sec:example}. 
Figure \ref{fig:example:secure} shows the secure version of the code given 
in Figure \ref{fig:example}. 
Line 3 imports the library API. 
Line 4 imports some sanitization functions. 
Line 6 marks command \texttt{os.system} (capable
to run arbitrary shell instructions) as a
sensitive sink to \texttt{OSI} attacks.
Tainted values reaching that sink must 
not contain the tag \texttt{OSI}.
%
Lines 7 and 8 establish 
that 
%indicates that those functions validate data and 
functions \texttt{s\_usermail} and \texttt{s\_file}
%, defined in 
%module \texttt{sanitize}, 
sanitize data in order to avoid \texttt{OSI}
attacks.
%Lines 7 and 8 indicates that those functions validate data and 
%remove tag \texttt{OSI} from tainted values. 
Lines 10 and 11 mark user input as untrustworthy.
When executing the program, the taint analysis  
raises an alarm on line 16.
The reason for that is that 
variable \texttt{cmd} is tainted with the tag \texttt{OSI}. 
Indeed, \texttt{cmd} is constructed from the untrustworthy values 
\texttt{usermail} and \texttt{file}. 
If we uncomment the lines where sanitization 
takes place (lines 12 and 13), the program runs normally, i.e. no
alarms are reported. 
Observe that the main part of the code (lines 14--16)  
are the same than in Figure \ref{fig:example}. 
%We argue that applying our library to other examples simply requires  
%to mark what are the sensitive sinks and sanitization functions (e.g.
%at the beginning of the code) as well as the sources of
%untrustworthy data. 

% 3 Pages
\section{Implementation}
\label{sec:implementation}
\begin{figure}[t]
{\small{
\lstinputlisting[language=Python,numbers=left, numberstyle=\tiny]{taint_class.py}
\caption{\label{fig:generate} Function to generate \nametklass classes}
}}
\end{figure}

In this section we present the details of our implementation. Due to lack of
space, we show the most interesting parts.
The full
implementation of the library is publicly available at \cite{PythonLib}.


%% Intuitively classes and methods
One of the core part of the library 
deals with how to keep track of taint information 
for built-in classes.
% and their corresponding operations.
%main challenge that the library implements is how to 
%The core of the library implements how built-in 
%values, and their corresponding operations, can keep track of 
%taint information. 
The library defines 
subclasses of built-in classes
in order to indicate if values are tainted or not.
%(like \texttt{str} or \texttt{int})
An object of these subclasses posses an attribute to indicate a set 
of tags 
%(Section \ref{sec:using}) 
associated to it. 
%When the set of tags is empty, objects are not tainted.
Objects are considered untainted when the
set of tags is empty. 
%Objects are not tainted when its set of tags is the empty.
%In the case that this set is empty, 
%the value is not tainted. 
We refer to these subclasses as
\emph{\nametklass classes}.
In addition, the methods inherited from the built-in classes 
are redefined in order to propagate taint information. 
%This propagation is done by updating the set of tags
%in tainted objects. 
More specifically, methods that belong to 
\nametklass classes return objects with  
the union of tags found in their arguments and 
the object calling the method. 
In Python, 
the dynamic dispatch 
mechanism guarantees that, for instance, 
the concatenations of untainted and tainted strings is performed 
%carried out 
with calls to methods of \nametklass classes, which properly   
%value, from a built-in class, and a tainted value, from its 
%corresponding \nametklass class, results in calls to 
%the methods of the \nametklass class. In that manner, 
%the 
propagates taint information. % can be successfully done. 
% can then be accordingly done.
%We illustrate these ideas later  with
%examples.

\subsection{Generating \nametklass classes}
\label{sec:general}

%\end{figure}
\begin{wrapfigure}{r}{7.5cm}
\vspace{-30pt}
{\small{
\lstinputlisting[language=Python,numbers=right, numberstyle=\tiny]{propagate.py}
\caption{\label{fig:propagate} Propagation of taint information}
}}
\vspace{-15pt}
\end{wrapfigure}
Figure \ref{fig:generate} presents a function to generate
\nametklass classes. The function takes a built-in
class (\texttt{klass}) and a list of its methods
(\texttt{methods}) where taint propagation 
must be performed. 
%\textbf{DESCRIBE ON THIS SENTENCE
%WHY NOT GENERAL INSTEAD OF A LIST OF METHODS}. 
Line 2 defines the name of the \nametklass class \texttt{tklass}.
Objects of \texttt{tklass}  
are associated to the empty set
 of tags when created (lines 3--6). Attribute 
\texttt{taints} is introduced to 
indicate the tags related to tainted values.
% to them.
Using Python's introspection features, variable 
\texttt{d} contains, among other
things, the list of methods for the built-in class (line 7). 
For each method in the built-in class and in \texttt{methods} 
(lines 8--10), the code adds to \texttt{tklass} a 
method that has the same name and computes the same results 
but also propagates taint information  
%, where the method 
%also performs propagation of taint information 
(line 11).
%Observe that added methods have the same names as the ones 
%in the built-in class.  
Function \texttt{propagate\_method} is explained below.
Lines 12--15 set method \texttt{\_\_radd\_\_} to taint-aware 
classes when built-in classes do not include that method but 
\texttt{\_\_add\_\_}.
%and not \texttt{\_\_radd\_\_}. 
Method \texttt{\_\_radd\_\_} is called to implement the binary operations 
with reflected (swapped) operands\footnote{The built-in class 
   for strings implements all 
   the reflected versions of its operators but \texttt{\_\_add\_\_}.}. 
%This function is called only if the left operand does not support the 
%corresponding operation. 
For instance, to evaluate the expression \texttt{x+y}, where \texttt{x} is a built-in string 
and y is a taint-aware string, Python calls \texttt{\_\_radd\_\_} from 
\texttt{y} and thus executing \texttt{y.\_\_radd\_\_(x)}. In that manner, 
the taint information of \texttt{y} is propagated to the expression. Otherwise, the method
\texttt{x.\_\_add\_\_(y)} is called instead, 
which results in an untainted string.
Finally, the \nametklass class is returned (line 16). 

The implementation of \texttt{propagate\_method} is shown in Figure
\ref{fig:propagate}. The function takes a method and returns another
method that computes the same results but propagates taint information. Line 3
calls the method received as argument and stores the results in 
\texttt{r}. 
Lines 4--9 collect the tags from the current object and 
the method's arguments into \texttt{t}. 
Variable \texttt{r} 
might refer to an object of a built-in class and
therefore not include the attribute \texttt{taints}. For that reason, 
function \texttt{taint\_aware} is designed to 
transform objects from built-in classes  
into \nametklass ones. 
For example, if \texttt{r} refers
to a list of objects of the class \texttt{str}, function \texttt{taint\_aware} returns  
a list of objects of the \nametklass class 
derived from \texttt{str}. 
Function \texttt{taint\_aware} 
is essentially implemented as a structural mapping on list, tuples,
sets, and dictionaries. 
The library does not
taint built-in containers, but rather their elements. This is a design decision 
based on the assumption that non-malicious code does not exploit
containers to circumvent the taint analysis (e.g. by 
encoding the value of tainted integers into 
the length of lists).
%For instance, we assume that nonmalicious programs do not 
\begin{wrapfigure}{r}{6.6cm}
\vspace{-20pt}
{\small{
\lstinputlisting[language=Python]{factory.py}
\caption{\label{fig:factory} \nameTklass classes for strings and
  integers}
}}
\vspace{-20pt}
\end{wrapfigure}
Otherwise, the implementation of the library can be easily adapted.
Line 11 returns the taint-aware version of \texttt{r} 
with the tags collected in \texttt{t}. 


%At the moment of implementing 
%\texttt{t\_}, some design decisions are made. 

  
To illustrate how to use  function \texttt{taint\_class}, Figure \ref{fig:factory} 
produces \nametklass classes for strings and integers, where 
\texttt{str\_methods} and \texttt{int\_methods} are lists 
of methods for the classes \texttt{str} and
\texttt{int}, respectively. Observe how the code presented in Figures
\ref{fig:generate} and \ref{fig:propagate} is general enough to be
applied to several built-in classes.
  

\subsection{Decorators}

\begin{wrapfigure}{r}{7.5cm}
\vspace{-30pt} 
{\small{
\lstinputlisting[language=Python,numbers=right, numberstyle=\tiny]{taint.py}
\caption{\label{fig:untrusted} Code for \texttt{untrusted}}
\vspace{-10pt}
}}
\vspace{-10pt}
\end{wrapfigure}
Except for \texttt{taint}, the rest of 
the API is implemented as decorators. In our library, 
decorators are high 
order functions \cite{IntroductionFunctional}, 
i.e. functions that take functions as arguments and return functions.
Figure \ref{fig:untrusted} shows the code for \texttt{untrusted}.
Function f, given as an argument, 
is the function that returns 
untrustworthy results (line 1). 
Intuitively, 
function \texttt{untrusted} returns
a function (\texttt{inner}) that 
calls function \texttt{f} (line 3) and taints the values 
returned by it (line 4). Symbol \texttt{TAGS} is the set of 
all the tags used by the library. 
% Due to lack of space, we only show
% the implementation of \texttt{untrusted}.
% Function \texttt{t\_} returns a taint-aware object when 
% \texttt{r} is an object of a built-in class. Otherwise, 
%  \texttt{t\_} updates the value of attribute \texttt{taints}.
Readers should refer to \cite{PythonLib} 
for the implementation details about the rest of the 
API.










\subsection{\nameTklass functions} 

\begin{wrapfigure}{r}{7.5cm}
\vspace{-30pt}
{\small{
\lstinputlisting[language=Python,numbers=right, numberstyle=\tiny]{propagate_func.py}
\vspace{-5pt}
\caption{\label{fig:propagate_func} Propagation of taint information among
  possibly different taint-aware objects}
}}
\end{wrapfigure}
Several dynamic taint analysis
\cite{Perl,Nguyen05,Jovanovic06pixy:a,KozlovPetukhov07,Futo07,SeoLam2010}
do not propagate taint information when results 
different from strings are computed from tainted values. 
(e.g. the length of a tainted string is usually an untainted
integer). This design decision might affect the abilities of taint
analysis to detect vulnerabilities. For instance, 
\begin{wrapfigure}{r}{7.5cm}
\vspace{-25pt}
{\small{
\lstinputlisting[language=Python]{example_connect.py}
\vspace{-5pt}
\caption{\label{fig:propagate_func:example} \nameTklass
  functions for strings and integers}
}}
\vspace{-20pt}
\end{wrapfigure}
taint analysis might miss dangerous patterns when 
programs encode strings as lists of numbers. 
A common workaround to this problem is to 
mark functions that perform
encodings of strings as sensitive sinks. In that manner, 
sanitization must occur before strings are represented in another
format. 
Nevertheless, this approach 
is unsatisfactory: the intrinsic meaning of sensitive sinks may be
lost. Sensitive sinks are security critical operations rather than 
functions that perform encodings of strings.
%It is not the encoding of strings what produces the 
%security breach, but rather their contents at 
%sensitive sinks.
%reaching security
%critical operations.
Our library provides means
to start breaching this gap. 

Figure \ref{fig:propagate_func} presents a general function that allows to define 
operations that return tainted values when their arguments 
involve taint-aware objects. As a result, it is possible
to define functions that, for instance, take tainted strings and 
return tainted integers. We classify this kind of functions 
as \emph{taint-aware}. 

Similar to the code shown in 
Figure \ref{fig:propagate}, \texttt{propagate\_func} is a high order
function. It takes function \texttt{f} 
and returns another function (\texttt{inner}) able to propagate 
taint information from the arguments
to the results.  Lines 3--7 collect tags 
from the arguments. If the set of
collected tags is empty, there are no tainted
values involved and therefore no taint propagation is 
performed (lines 9--10). Otherwise, 
a taint-aware version of the results is returned with 
the tags collected in the arguments (line 11). 
%with the attribute \texttt{taints} sets to 
%the union of the tags found in the arguments (line 12). 
 
To illustrate the use of \texttt{propagate\_func}, Figure
\ref{fig:propagate_func:example} shows 
some \nametklass functions 
for strings and integers. We redefine the standard functions to 
compute lengths of lists (\texttt{len}), the ASCII code of a character
(\texttt{chr}), and its inverse (\texttt{ord}). As a result, 
\texttt{len(taint('string'))} returns the tainted integer 6.
It is up to the users 
of the library to decide which functions must be \nametklass depending
on the scenario.
The library only provides 
%the means for that 
%(i.e. function \texttt{propagate\_func}) and some 
redefinition of standard functions 
like the ones 
shown in Figure \ref{fig:propagate_func:example}.
% % %


\subsection{Scope of the library}

In Figure \ref{fig:generate}, 
the method to automatically produce \nametklass classes does not 
work with booleans. 
The reason for that is that class \texttt{bool}
cannot be subclassed in Python 
\footnote{\url{http://docs.python.org/library/functions.html#bool}}.
%This restriction seems to be an arbitrary decision and we could 
%not find technical reasons to justify it. 
Consequently, our library
cannot handle tainted boolean values. We argue that 
this shortcoming does not restrict the usability of the library for
 two reasons. Firstly, different from previous approaches 
\cite{Perl,Nguyen05,Jovanovic06pixy:a,KozlovPetukhov07,Futo07,SeoLam2010}, 
the library can provide taint analysis for several 
built-in types rather than just strings. Secondly, we consider that 
booleans are typically used on guards. 
Since the library already ignores implicit flows, 
the possibilities to find vulnerabilities
are not drastically reduced by disregarding taint information on booleans.
% Not sure about this:
% Moreover, implicit 
% flows can be produced by having boolean expressions and 
% short-circuit evaluation as the 
% semantics for boolean operators. For instance, An implicit 
% flow can be represented as the boolean expression
% \texttt{(not ( x == taint('a') ) or 'a')}. 
% This expression evaluates to \texttt{True} if
% \texttt{x == 'a'}. Otherwise, it evaluates to \texttt{'a'}.

When generating the taint-aware
class \texttt{STR} (Figure \ref{fig:factory}), 
we found some problems 
when dealing with   
%we discover that the Python interpreter
%\footnote{CPython, version 2.6.x} forbids 
%to redefine 
some methods from the class \texttt{str}. 
Python interpreter raises exceptions when  
methods 
\texttt{\_\_nonzero\_\_},
\texttt{\_\_reduce\_\_}, and 
\texttt{\_\_reduce\_ex\_\_} are redefined.
Moreover, 
when methods  
\texttt{\_\_new\_\_}, \texttt{\_\_init\_\_}, 
\texttt{\_\_getattribute\_\_},  
%\texttt{\_\_cmp\_\_},  
%\texttt{\_\_nonzero\_\_},
and 
\texttt{\_\_repr\_\_}
%\texttt{\_\_reduce\_\_}, and 
%\texttt{\_\_reduce\_ex\_\_}. 
are redefined by function \texttt{taint\_class},
an infinite recursion is produced when calling any of them.
As for \texttt{STR}, the generation of the \nametklass 
class \texttt{INT} exposes the same behavior, i.e. the 
methods mentioned before produce the same problems.
We argue 
that this restriction does not drastically impact on the capabilities to 
detect vulnerabilities.
%Methods named among underscores  
%are sinvoked by special syntax rather than explicitly 
%\footnote{\url{http://docs.python.org/reference/datamodel.html#specialnames}}. 
Methods \texttt{\_\_new\_\_} 
is called 
when creating objects.
In Figure \ref{fig:generate}, \nametklass 
classes define this method 
on line 3.
Method \texttt{\_\_init\_\_} is called when 
initializing objects.  
Python invokes this method  after an object is created 
and programs do not usually called it explicitly.
Method \texttt{\_\_getattribute\_\_}
is used to access any attribute on a class.
This method is automatically inherited from 
\texttt{klass} and it works as expected for 
\nametklass classes.
%defined by 
%Python when setting up classes.
%Thus, 
%method \texttt{\_\_getattribute\_\_} 
%is defined for  
%\texttt{STR} when the class is created.
%Method \texttt{\_\_cmp\_\_} is applied to compare
%objects. 
Method \texttt{\_\_nonzero\_\_} is called when 
objects need to be converted into a boolean value.
%Similarly to bools, 
As mentioned before, the
analysis ignores taint information of data 
that is typically used on guards. 
Method \texttt{\_\_repr\_\_} pretty prints  
objects on the screen. 
In principle, developers should be careful 
to not use calls to \texttt{\_\_repr\_\_} in order to 
convert tainted objects into untainted ones. 
However, this method is typically used for debugging 
\footnote{\url{http://docs.python.org/reference/datamodel.html}}.
Methods \texttt{\_\_reduce\_\_} and \texttt{\_\_reduce\_ex\_\_}
are used by Pickle \footnote{An special Python module} to serialize strings. 
Given these facts, 
the argument \texttt{method} in function \texttt{taint\_class} 
establishes the methods to be redefined on taint-aware classes
(Figure \ref{fig:generate}).
This argument is also useful when
the built-in classes might 
vary among different Python interpreters.
%different versions of Python differ on the methods provided by 
%the built-in classes .
It is future work 
to automatically determine the lists of methods to be redefined for different 
built-in classes and different versions of Python.

%% Connecting functions
It is up to the users of the library 
to decide which built-in classes and functions must be
taint-aware. This attitude comes from the need of being flexible 
and not affecting performance unless it is necessary. 
Why users interested on 
taint analysis for strings should accept 
run-time overheads due to tainted integers? 


%%% Taint marks lost!
It is important to remark that the library only tracks taint information 
in the source code being developed. As a consequence, 
taint information could be lost if, for example, 
taint values are given to external 
libraries (or libraries written in other languages)
that are not taint-aware. 
One way to tackle this problem is to augment the library 
functions to be taint-aware by applying \texttt{propagate\_func} to them. 
 

As a future work, we will explore if 
%defining 
%some \nametklass classes, 
it is possible to automatically 
define \namefunc functions 
based on the built-in functions (found in the 
interpreter) and \nametklass classes in order to 
increase the number of \nametklass functions provided by the library.
At the moment, the library provides 
\nametklass classes for
strings, integers, floats, and unicode as well as some 
\namefunc functions (e.g. \texttt{len}, \texttt{chr}, 
and \texttt{ord}). 



% 1 Page
\section{Related Work}
\label{sec:related}

%% Intro
A considerable amount of literature has been published on 
taint analysis. Readers can refer to 
\cite{Chang:et:al:CCS08} for a description 
of how this technique has been applied 
on different research areas. In this section, we focus on 
analyses developed for popular web scripting languages. 



%% Native support for taint modes. Perl and Ruby
Perl \cite{Perl} was the first scripting language to include taint analysis
as a native feature of the interpreter. Perl taint mode marks strings
originated from outside a program as tainted (i.e. 
inputs from users, environment variables, and files). Sanitization
is done by using regular expressions. Writing to files, 
executing shell commands, and sending information over the network are
considered sensitive sinks. Differently, 
our library gives freedom to developers 
to classify the sources of 
tainted data, sanitization functions, and sensitive sinks. 
Similar to Perl, Ruby \cite{thomas2004prub} provides 
support for taint analysis. Ruby's taint mode, however, 
performs analysis at the level of
objects rather than only strings. Both, Perl and Ruby, utilize dynamic
techniques for their analyses.

%% PHP 
Several taint analysis have been developed for the popular
scripting language PHP. 
Aiming to avoid any user intervention, 
authors in \cite{WebSSARI} combine static and dynamic
techniques to automatically repair vulnerabilities in PHP code. 
They 
%develop a type-system to detect where to 
%insert some predetermined 
%sanitization functions when tainted values
propose to use static analysis (type-system) in order to insert some predetermined 
sanitization functions when tainted values  
reach sensitive sinks. 
%Sanitization functions
%are then executed at run-time. 
Observe that the semantic 
of programs might be changed when inserting calls to sanitization functions, which constitutes the dynamic
part of the analysis in \cite{WebSSARI}.  
Our approach, on the other hand, does not implement a type-system and only reports vulnerabilities, i.e.  
it is up to developers
to decide where, and how, sanitization procedures must be called. 
In \cite{Nguyen05}, 
Nguyen-Toung et al. adapt the PHP interpreter to provide a dynamic 
taint analysis at the level of characters, which the authors 
call \emph{precise tainting}. 
They argue that precise tainting gains precision over traditional 
taint analyses for strings.
%and thus reducing the number of 
%false alarms. 
Authors need to manually  
exploit, when feasible, semantics definitions of functions 
in order to accurately keep track of tainted characters. 
Our approach, on the other hand, uses the same mechanism to 
handle tainted values 
independently of the nature of a given function. Consequently, 
we are able to automatically extend our analysis to 
different set of data types but without being as precise as 
Nguyen-Toung et al.' work. 
It is worth seeing studies indicating 
how much precision (i.e. less false alarms) 
it is obtained with \emph{precise tainting} in practice.
Similarly to Nguyen-Toung et al.'s work,
Futoransky \cite{Futo07} 
et al. provide a precise dynamic taint analysis for PHP.
Pietraszek and Berghe \cite{Pietraszek05defendingagainst}
modify the PHP runtime environment to 
assign \emph{metadata} to user-provided input as well as to 
provide metadata-preserving string operations. Security critical 
operations are also   
instrumented to evaluate, when taken strings as input, 
the risk of executing such operations based on the assigned metadata.
Jovanovic et al. \cite{Jovanovic06pixy:a} 
propose to combine a traditional data flow and alias 
analysis to increase the precision of their static taint analysis for PHP.
They observe a $50\%$ rate of false alarms (i.e. one false alarm for 
each vulnerability). 
The works in \cite{Saner, Monga:et:all:2009} combine static 
and dynamic techniques.
The static techniques are 
%essentially 
used to 
reduce the number of program variables 
where taint information must be tracked at 
run-time.


%% Java
A taint analysis for Java \cite{Haldar05dynamictaint} instruments
the class \texttt{java.lang.String} as well as  classes that 
present untrustworthy sources and sensitive sinks. 
The instrumentation of \texttt{java.lang.String} is done offline, 
while other classes are instrumented online. The authors 
mention that a custom class loader in the JVM is needed in order 
to perform online instrumentation. Another taint analysis 
for Java \cite{TrippPFSW09}, called TAJ, focus on scalability and 
performance requirements for industry-level applications.
To achieve industrial demands, TAJ uses
static tecniques for pointer analysis, call-graph construction, 
and slicing. Similarly, the authors in 
\cite{Livshits:Lam:USENIX} propose an static analysis 
for Java that focus on achieving precision and scalability.
 


%We obtain a similar effect than online instrumentation 
%due to decorators, class inheritance, and overloading in Python.

%% Information-flow as a library
A series of work 
\cite{PengLi+:Arrows,Tsai+:Library,Russo+:Haskell08} propose
to provide information-flow security via a library in Haskell. 
These libraries handle explicit and implicit flows and programmers
need to write programs with an special-purpose API.
Similar to other taint analysis, 
our library does not contemplate implicit flows %(see Section \ref{sec:library})
and programs do not need to be written 
with an special-purpose API. 

Among the closest related work, we can mention 
\cite{KozlovPetukhov07} and \cite{SeoLam2010}.
In  \cite{KozlovPetukhov07}, authors modify the Python interpreter 
to provide a dynamic taint analysis. More specifically, %they extend
the representation of the class \texttt{str} is extended to include a boolean 
flag to indicate if a string is tainted. We provide a similar analysis 
%the same 
%taint analysis 
but without modifying the interpreter.
The work by Seo and Lam \cite{SeoLam2010}, called InvisiType, 
aims to enforce safety checks without 
modifying the analyzed code. Similar to our assumptions, their approach 
is designed to work with non-malicious code. InvisiType is more general 
than our approach. In fact, authors show how InvisiType can provide taint
analysis and access control checks for Python programs. However, 
InvisiType relies on several modifications in the Python interpreter 
in order to perform the security checks at the right places. 
For example, when native methods are called, 
the run-time environment firstly calls the special purpose 
method \texttt{\_\_nativecall\_\_}.  
%to 
%propagate taint information. 
As a manner to specifying policies, 
the approach provides the class \texttt{InvisiType} that defines special 
purposes methods to get support from the run-time system (e.g.
\texttt{\_\_nativecall\_\_} is one of those methods). 
Subclasses of this class represent security policies. 
The approach relies on multiple inheritance to 
extend existing classes with security checks. 
To include or remove security checks from objects, 
programs need to explicitly call functions \emph{promote} and 
\emph{demote}. Being less invasive, our library uses 
decorators instead of explicit function calls to taint and
untaint data. 
%Due to its specific purpose and simplicity, 
Our approach does not require 
multiple inheritance. 

% 1/2 Page
\section{Conclusions}
\label{sec:conclusions}

We propose a taint mode for Python via a library entirely written
in Python.
We show that no modifications in the interpreter are needed. 
%To the best of our knowledge, 
%this is the first library to provide a taint mode for a web 
%scripting language. 
Different from traditional taint analysis, our library
is able to 
keep track of tainted values for  
several built-in classes.
Additionally, the library provide means to define functions that 
propagate taint information
%from the arguments to the results 
(e.g. the length of a tainted string produces a
tainted integer). The library consists on around 300 LOC.
To apply taint analysis in programs, it is only needed 
to indicate the sources of untrustworthy data, sensitive 
sinks, and sanitization functions. The library uses decorators 
as a noninvasive approach to mark source code. 
Python's object classes 
and dynamic dispatch mechanism allow the analysis to be executed with almost no modifications
in the  code. %Our approach seems  promising.
%to detect vulnerabilities in real 
%applications. 
As a future work, we plan to use the library to harden frameworks
for web development and evaluate the capabilities of our library to 
detect vulnerabilities
in popular web applications.


<<<<<<< main.tex

%{\small{
%\paragraph{Acknowledgments} Anonimyzed
%Thanks are due to Arnar Birgisson for the interesting discussions.
%This work was funded by the Swedish research agencies VR and 
%the scholarship program for graduated students 
%from the Universidad Tecnol\'{o}gica Nacional. 
%}}
=======
{\small{
\paragraph{Acknowledgments}
Thanks are due to Arnar Birgisson for interesting discussions.
This work was funded by the Swedish research agencies VR and 
the scholarship program for graduated students 
from the Universidad Tecnol\'{o}gica Nacional, Facultad Regional Santa Fe. 
}}
>>>>>>> 1.56
{\small{
\bibliographystyle{abbrv}
\bibliography{literature}
}}


\end{document}

 