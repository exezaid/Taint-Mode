A way to face these problems is using Taint Analysis. Usually enforcesd 

Data received from a client is considerer tainted. We can't trust in data from the outside because we don't know who generate it. May be a real user, maybe an attacker or even an attacker program. 

Tainted data can be untainted by a sanitization process.

We don't want tainted data to reach sensitive sinks.

In the image you can see different sanitization processes represented with different colors. This means that data that will finish in different sinks needs to be properly cleaned for that kind of sink.

It's not the same the function you'll use to protect a page renderer against XSS than the DB against SQLI.

% Taint analsys
%A 
Taint analysis is an automatic approach to find vulnerabilities.
Intuitively, taint analysis restricts how tainted or untrustworthy 
data flow inside programs. Specifically, it constrains data 
to be untainted (trustworthy) 
or previously sanitized when 
reaching sensitive sinks. 
%In a web scenario, 
%user input is frequently considered as tainted, while 
%HTTP-request, SQL queries, and other security critical operations 
%are classified as sensitive sinks. 
Perl was the first scripting 
language to provide taint analysis 
as an special mode of the  
interpreter called \emph{taint mode} \cite{BekmanCholet2003}. 
Similar to Perl, some interpreters for 
 Ruby \cite{thomas2004prub}, PHP \cite{Nguyen05}, and 
recently Python \cite{KozlovPetukhov07} have been 
carefully modified to provide taint modes.
Adapting interpreters to incorporate taint analysis 
present two major drawbacks that directly 
impact on the adoption of this technology. 
Firstly, incorporating taint analysis into an interpreter 
might be a major task in its own right. Secondly, it is 
very probably that it is necessary to repeatedly adapt an  
interpreter at every new version or release of it.


\begin{wrapfigure}{r}{6.5cm}
%\begin{figure}[t]
\vspace{-25pt}
{\small{
\lstinputlisting[language=Python,numbers=right, numberstyle=\tiny]{email.py}
\caption{\label{fig:example}Code for \texttt{email.py}}
}}
\vspace{-15pt}
%\end{figure}
\end{wrapfigure}
%%% Solution / Contributions
Rather than modifying interpreters, we present how to provide
a taint mode for Python via a library written entirely in Python. 
Python is spreading fast inside
web development \cite{WikiPython}. 
%For instance, 
%the popular Wiki engine 
%MoinMoin (e.g. used by Ubuntu forums)
%and Youtube are mostly written in Python.  Among companies, 
%we can mention Google, which uses 
%the language for some of its search-engine internals at
%Google Groups, Gmail, and Google Maps.
Besides its successful use, Python presents 
some programming languages abstractions that makes possible 
to provide a taint mode via a library. For example, 
Python decorators \cite{Lutz:1999:LP} are a non-invasive and simple 
manner to declare sources of tainted data, sensitive sinks, and 
sanitation functions. Python's 
object-oriented and dynamic typing mechanisms allows the 
execution of the taint analysis with almost no modifications in the
source code. 
%\marginpar{Check later what about decorating Bool} 


The library provides a general method to enhance Python's built-in
classes with tainted values. 
%To demostrate the flexibility of our approach, 
%the library enhances the built-in classes for
%strings, integers, and floats. 
In general, taint
analysis tends to only consider strings or characters 
\cite{Perl,Nguyen05,Haldar05dynamictaint,KozlovPetukhov07,Futo07,SeoLam2010}.
% when 
%implemented as part of interpreters. 
In contrast, our library 
can be easily adapted to consider different built-in
classes
% as well as users' specific ones, 
and thus providing a taint
analysis for a wider set of data types. 
By only considering tainted strings, the library provides 
a similar analysis than in \cite{KozlovPetukhov07},
but without modifying the Python interpreter.
To the best of our knowledge, a
library for taint analysis has not been considered before. 

\section{Related work}
