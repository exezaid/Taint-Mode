\begin{figure}[t]
{\small{
\lstinputlisting[language=Python,numbers=left, numberstyle=\tiny]{taint_class.py}
\caption{\label{fig:generate} Function to generate \nametklass classes}
}}
\end{figure}

In this section we present the details of our implementation. Due to lack of
space, we show the most interesting parts.
The full
implementation of the library is publicly available at \cite{PythonLib}.


%% Intuitively classes and methods
One of the core part of the library 
deals with how to keep track of taint information 
for built-in classes.
% and their corresponding operations.
%main challenge that the library implements is how to 
%The core of the library implements how built-in 
%values, and their corresponding operations, can keep track of 
%taint information. 
The library defines 
subclasses of built-in classes
in order to indicate if values are tainted or not.
%(like \texttt{str} or \texttt{int})
An object of these subclasses posses an attribute to indicate a set 
of tags 
%(Section \ref{sec:using}) 
associated to it. 
%When the set of tags is empty, objects are not tainted.
Objects are considered untainted when the
set of tags is empty. 
%Objects are not tainted when its set of tags is the empty.
%In the case that this set is empty, 
%the value is not tainted. 
We refer to these subclasses as
\emph{\nametklass classes}.
In addition, the methods inherited from the built-in classes 
are redefined in order to propagate taint information. 
%This propagation is done by updating the set of tags
%in tainted objects. 
More specifically, methods that belong to 
\nametklass classes return objects with  
the union of tags found in their arguments and 
the object calling the method. 
In Python, 
the dynamic dispatch 
mechanism guarantees that, for instance, 
the concatenations of untainted and tainted strings is performed 
%carried out 
with calls to methods of \nametklass classes, which properly   
%value, from a built-in class, and a tainted value, from its 
%corresponding \nametklass class, results in calls to 
%the methods of the \nametklass class. In that manner, 
%the 
propagates taint information. % can be successfully done. 
% can then be accordingly done.
%We illustrate these ideas later  with
%examples.

\section{Generating \nametklass classes}
\label{sec:general}

%\end{figure}
\begin{wrapfigure}{r}{7.5cm}
\vspace{-30pt}
{\small{
\lstinputlisting[language=Python,numbers=right, numberstyle=\tiny]{propagate.py}
\caption{\label{fig:propagate} Propagation of taint information}
}}
\vspace{-15pt}
\end{wrapfigure}
Figure \ref{fig:generate} presents a function to generate
\nametklass classes. The function takes a built-in
class (\texttt{klass}) and a list of its methods
(\texttt{methods}) where taint propagation 
must be performed. 
%\textbf{DESCRIBE ON THIS SENTENCE
%WHY NOT GENERAL INSTEAD OF A LIST OF METHODS}. 
Line 2 defines the name of the \nametklass class \texttt{tklass}.
Objects of \texttt{tklass}  
are associated to the empty set
 of tags when created (lines 3--6). Attribute 
\texttt{taints} is introduced to 
indicate the tags related to tainted values.
% to them.
Using Python's introspection features, variable 
\texttt{d} contains, among other
things, the list of methods for the built-in class (line 7). 
For each method in the built-in class and in \texttt{methods} 
(lines 8--10), the code adds to \texttt{tklass} a 
method that has the same name and computes the same results 
but also propagates taint information  
%, where the method 
%also performs propagation of taint information 
(line 11).
%Observe that added methods have the same names as the ones 
%in the built-in class.  
Function \texttt{propagate\_method} is explained below.
Lines 12--15 set method \texttt{\_\_radd\_\_} to taint-aware 
classes when built-in classes do not include that method but 
\texttt{\_\_add\_\_}.
%and not \texttt{\_\_radd\_\_}. 
Method \texttt{\_\_radd\_\_} is called to implement the binary operations 
with reflected (swapped) operands\footnote{The built-in class 
   for strings implements all 
   the reflected versions of its operators but \texttt{\_\_add\_\_}.}. 
%This function is called only if the left operand does not support the 
%corresponding operation. 
For instance, to evaluate the expression \texttt{x+y}, where \texttt{x} is a built-in string 
and y is a taint-aware string, Python calls \texttt{\_\_radd\_\_} from 
\texttt{y} and thus executing \texttt{y.\_\_radd\_\_(x)}. In that manner, 
the taint information of \texttt{y} is propagated to the expression. Otherwise, the method
\texttt{x.\_\_add\_\_(y)} is called instead, 
which results in an untainted string.
Finally, the \nametklass class is returned (line 16). 

The implementation of \texttt{propagate\_method} is shown in Figure
\ref{fig:propagate}. The function takes a method and returns another
method that computes the same results but propagates taint information. Line 3
calls the method received as argument and stores the results in 
\texttt{r}. 
Lines 4--9 collect the tags from the current object and 
the method's arguments into \texttt{t}. 
Variable \texttt{r} 
might refer to an object of a built-in class and
therefore not include the attribute \texttt{taints}. For that reason, 
function \texttt{taint\_aware} is designed to 
transform objects from built-in classes  
into \nametklass ones. 
For example, if \texttt{r} refers
to a list of objects of the class \texttt{str}, function \texttt{taint\_aware} returns  
a list of objects of the \nametklass class 
derived from \texttt{str}. 
Function \texttt{taint\_aware} 
is essentially implemented as a structural mapping on list, tuples,
sets, and dictionaries. 
The library does not
taint built-in containers, but rather their elements. This is a design decision 
based on the assumption that non-malicious code does not exploit
containers to circumvent the taint analysis (e.g. by 
encoding the value of tainted integers into 
the length of lists).
%For instance, we assume that nonmalicious programs do not 
\begin{wrapfigure}{r}{6.6cm}
\vspace{-20pt}
{\small{
\lstinputlisting[language=Python]{factory.py}
\caption{\label{fig:factory} \nameTklass classes for strings and
  integers}
}}
\vspace{-20pt}
\end{wrapfigure}
Otherwise, the implementation of the library can be easily adapted.
Line 11 returns the taint-aware version of \texttt{r} 
with the tags collected in \texttt{t}. 


%At the moment of implementing 
%\texttt{t\_}, some design decisions are made. 

  
To illustrate how to use  function \texttt{taint\_class}, Figure \ref{fig:factory} 
produces \nametklass classes for strings and integers, where 
\texttt{str\_methods} and \texttt{int\_methods} are lists 
of methods for the classes \texttt{str} and
\texttt{int}, respectively. Observe how the code presented in Figures
\ref{fig:generate} and \ref{fig:propagate} is general enough to be
applied to several built-in classes.
  

\section{Decorators}

\begin{wrapfigure}{r}{7.5cm}
\vspace{-30pt} 
{\small{
\lstinputlisting[language=Python,numbers=right, numberstyle=\tiny]{taint.py}
\caption{\label{fig:untrusted} Code for \texttt{untrusted}}
\vspace{-10pt}
}}
\vspace{-10pt}
\end{wrapfigure}
Except for \texttt{taint}, the rest of 
the API is implemented as decorators. In our library, 
decorators are high 
order functions \cite{IntroductionFunctional}, 
i.e. functions that take functions as arguments and return functions.
Figure \ref{fig:untrusted} shows the code for \texttt{untrusted}.
Function f, given as an argument, 
is the function that returns 
untrustworthy results (line 1). 
Intuitively, 
function \texttt{untrusted} returns
a function (\texttt{inner}) that 
calls function \texttt{f} (line 3) and taints the values 
returned by it (line 4). Symbol \texttt{TAGS} is the set of 
all the tags used by the library. 
% Due to lack of space, we only show
% the implementation of \texttt{untrusted}.
% Function \texttt{t\_} returns a taint-aware object when 
% \texttt{r} is an object of a built-in class. Otherwise, 
%  \texttt{t\_} updates the value of attribute \texttt{taints}.
Readers should refer to \cite{PythonLib} 
for the implementation details about the rest of the 
API.

\section{\nameTklass functions} 

\begin{wrapfigure}{r}{7.5cm}
\vspace{-30pt}
{\small{
\lstinputlisting[language=Python,numbers=right, numberstyle=\tiny]{propagate_func.py}
\vspace{-5pt}
\caption{\label{fig:propagate_func} Propagation of taint information among
  possibly different taint-aware objects}
}}
\end{wrapfigure}
Several dynamic taint analysis
\cite{Perl,Nguyen05,Jovanovic06pixy:a,KozlovPetukhov07,Futo07,SeoLam2010}
do not propagate taint information when results 
different from strings are computed from tainted values. 
(e.g. the length of a tainted string is usually an untainted
integer). This design decision might affect the abilities of taint
analysis to detect vulnerabilities. For instance, 
\begin{wrapfigure}{r}{7.5cm}
\vspace{-25pt}
{\small{
\lstinputlisting[language=Python]{example_connect.py}
\vspace{-5pt}
\caption{\label{fig:propagate_func:example} \nameTklass
  functions for strings and integers}
}}
\vspace{-20pt}
\end{wrapfigure}
taint analysis might miss dangerous patterns when 
programs encode strings as lists of numbers. 
A common workaround to this problem is to 
mark functions that perform
encodings of strings as sensitive sinks. In that manner, 
sanitization must occur before strings are represented in another
format. 
Nevertheless, this approach 
is unsatisfactory: the intrinsic meaning of sensitive sinks may be
lost. Sensitive sinks are security critical operations rather than 
functions that perform encodings of strings.
%It is not the encoding of strings what produces the 
%security breach, but rather their contents at 
%sensitive sinks.
%reaching security
%critical operations.
Our library provides means
to start breaching this gap. 

Figure \ref{fig:propagate_func} presents a general function that allows to define 
operations that return tainted values when their arguments 
involve taint-aware objects. As a result, it is possible
to define functions that, for instance, take tainted strings and 
return tainted integers. We classify this kind of functions 
as \emph{taint-aware}. 

Similar to the code shown in 
Figure \ref{fig:propagate}, \texttt{propagate\_func} is a high order
function. It takes function \texttt{f} 
and returns another function (\texttt{inner}) able to propagate 
taint information from the arguments
to the results.  Lines 3--7 collect tags 
from the arguments. If the set of
collected tags is empty, there are no tainted
values involved and therefore no taint propagation is 
performed (lines 9--10). Otherwise, 
a taint-aware version of the results is returned with 
the tags collected in the arguments (line 11). 
%with the attribute \texttt{taints} sets to 
%the union of the tags found in the arguments (line 12). 
 
To illustrate the use of \texttt{propagate\_func}, Figure
\ref{fig:propagate_func:example} shows 
some \nametklass functions 
for strings and integers. We redefine the standard functions to 
compute lengths of lists (\texttt{len}), the ASCII code of a character
(\texttt{chr}), and its inverse (\texttt{ord}). As a result, 
\texttt{len(taint('string'))} returns the tainted integer 6.
It is up to the users 
of the library to decide which functions must be \nametklass depending
on the scenario.
The library only provides 
%the means for that 
%(i.e. function \texttt{propagate\_func}) and some 
redefinition of standard functions 
like the ones 
shown in Figure \ref{fig:propagate_func:example}.
% % %

