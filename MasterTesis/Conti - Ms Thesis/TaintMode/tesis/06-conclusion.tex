We propose a taint mode for Python via a library entirely written
in Python.
We show that no modifications in the interpreter are needed. 
%To the best of our knowledge, 
%this is the first library to provide a taint mode for a web 
%scripting language. 
Different from traditional taint analysis, our library
is able to 
keep track of tainted values for  
several built-in classes.
Additionally, the library provide means to define functions that 
propagate taint information
%from the arguments to the results 
(e.g. the length of a tainted string produces a
tainted integer). The library consists on around 300 LOC.
To apply taint analysis in programs, it is only needed 
to indicate the sources of untrustworthy data, sensitive 
sinks, and sanitization functions. The library uses decorators 
as a noninvasive approach to mark source code. 
Python's object classes 
and dynamic dispatch mechanism allow the analysis to be executed with almost no modifications
in the  code. %Our approach seems  promising.
%to detect vulnerabilities in real 
%applications. 
As a future work, we plan to use the library to harden frameworks
for web development and evaluate the capabilities of our library to 
detect vulnerabilities
in popular web applications.
