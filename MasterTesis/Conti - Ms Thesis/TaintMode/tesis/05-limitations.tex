
\section{Scope of the library}

In Figure \ref{fig:generate}, 
the method to automatically produce \nametklass classes does not 
work with booleans. 
The reason for that is that class \texttt{bool}
cannot be subclassed in Python 
\footnote{\url{http://docs.python.org/library/functions.html#bool}}.
%This restriction seems to be an arbitrary decision and we could 
%not find technical reasons to justify it. 
Consequently, our library
cannot handle tainted boolean values. We argue that 
this shortcoming does not restrict the usability of the library for
 two reasons. Firstly, different from previous approaches 
\cite{Perl,Nguyen05,Jovanovic06pixy:a,KozlovPetukhov07,Futo07,SeoLam2010}, 
the library can provide taint analysis for several 
built-in types rather than just strings. Secondly, we consider that 
booleans are typically used on guards. 
Since the library already ignores implicit flows, 
the possibilities to find vulnerabilities
are not drastically reduced by disregarding taint information on booleans.
% Not sure about this:
% Moreover, implicit 
% flows can be produced by having boolean expressions and 
% short-circuit evaluation as the 
% semantics for boolean operators. For instance, An implicit 
% flow can be represented as the boolean expression
% \texttt{(not ( x == taint('a') ) or 'a')}. 
% This expression evaluates to \texttt{True} if
% \texttt{x == 'a'}. Otherwise, it evaluates to \texttt{'a'}.

When generating the taint-aware
class \texttt{STR} (Figure \ref{fig:factory}), 
we found some problems 
when dealing with   
%we discover that the Python interpreter
%\footnote{CPython, version 2.6.x} forbids 
%to redefine 
some methods from the class \texttt{str}. 
Python interpreter raises exceptions when  
methods 
\texttt{\_\_nonzero\_\_},
\texttt{\_\_reduce\_\_}, and 
\texttt{\_\_reduce\_ex\_\_} are redefined.
Moreover, 
when methods  
\texttt{\_\_new\_\_}, \texttt{\_\_init\_\_}, 
\texttt{\_\_getattribute\_\_},  
%\texttt{\_\_cmp\_\_},  
%\texttt{\_\_nonzero\_\_},
and 
\texttt{\_\_repr\_\_}
%\texttt{\_\_reduce\_\_}, and 
%\texttt{\_\_reduce\_ex\_\_}. 
are redefined by function \texttt{taint\_class},
an infinite recursion is produced when calling any of them.
As for \texttt{STR}, the generation of the \nametklass 
class \texttt{INT} exposes the same behavior, i.e. the 
methods mentioned before produce the same problems.
We argue 
that this restriction does not drastically impact on the capabilities to 
detect vulnerabilities.
%Methods named among underscores  
%are sinvoked by special syntax rather than explicitly 
%\footnote{\url{http://docs.python.org/reference/datamodel.html#specialnames}}. 
Methods \texttt{\_\_new\_\_} 
is called 
when creating objects.
In Figure \ref{fig:generate}, \nametklass 
classes define this method 
on line 3.
Method \texttt{\_\_init\_\_} is called when 
initializing objects.  
Python invokes this method  after an object is created 
and programs do not usually called it explicitly.
Method \texttt{\_\_getattribute\_\_}
is used to access any attribute on a class.
This method is automatically inherited from 
\texttt{klass} and it works as expected for 
\nametklass classes.
%defined by 
%Python when setting up classes.
%Thus, 
%method \texttt{\_\_getattribute\_\_} 
%is defined for  
%\texttt{STR} when the class is created.
%Method \texttt{\_\_cmp\_\_} is applied to compare
%objects. 
Method \texttt{\_\_nonzero\_\_} is called when 
objects need to be converted into a boolean value.
%Similarly to bools, 
As mentioned before, the
analysis ignores taint information of data 
that is typically used on guards. 
Method \texttt{\_\_repr\_\_} pretty prints  
objects on the screen. 
In principle, developers should be careful 
to not use calls to \texttt{\_\_repr\_\_} in order to 
convert tainted objects into untainted ones. 
However, this method is typically used for debugging 
\footnote{\url{http://docs.python.org/reference/datamodel.html}}.
Methods \texttt{\_\_reduce\_\_} and \texttt{\_\_reduce\_ex\_\_}
are used by Pickle \footnote{An special Python module} to serialize strings. 
Given these facts, 
the argument \texttt{method} in function \texttt{taint\_class} 
establishes the methods to be redefined on taint-aware classes
(Figure \ref{fig:generate}).
This argument is also useful when
the built-in classes might 
vary among different Python interpreters.
%different versions of Python differ on the methods provided by 
%the built-in classes .
It is future work 
to automatically determine the lists of methods to be redefined for different 
built-in classes and different versions of Python.

%% Connecting functions
It is up to the users of the library 
to decide which built-in classes and functions must be
taint-aware. This attitude comes from the need of being flexible 
and not affecting performance unless it is necessary. 
Why users interested on 
taint analysis for strings should accept 
run-time overheads due to tainted integers? 


%%% Taint marks lost!
It is important to remark that the library only tracks taint information 
in the source code being developed. As a consequence, 
taint information could be lost if, for example, 
taint values are given to external 
libraries (or libraries written in other languages)
that are not taint-aware. 
One way to tackle this problem is to augment the library 
functions to be taint-aware by applying \texttt{propagate\_func} to them. 
 

As a future work, we will explore if 
%defining 
%some \nametklass classes, 
it is possible to automatically 
define \namefunc functions 
based on the built-in functions (found in the 
interpreter) and \nametklass classes in order to 
increase the number of \nametklass functions provided by the library.
At the moment, the library provides 
\nametklass classes for
strings, integers, floats, and unicode as well as some 
\namefunc functions (e.g. \texttt{len}, \texttt{chr}, 
and \texttt{ord}). 

\section{The \% operator problem}
