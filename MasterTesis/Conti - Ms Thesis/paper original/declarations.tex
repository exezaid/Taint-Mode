% Time-stamp: <2006-08-24 16:37:12 Andrei Sabelfeld> 


\newcommand{\prefix}[2]{\ensuremath{\mathit{prefix}(#1,#2)}}
\newcommand{\foreverhigh}{\ensuremath{\Rightarrow_\high}}


\newcommand{\RuleSideCond}[4]
{\centerline{
   { \ensuremath{
           \mprset{flushleft}
           \inferrule*[Left=#1, Right=#4]
           {#2}
           {#3}
         }
   }
}}

\newcommand{\spp}{\hspace{2pt}} 
\newcommand{\hh}{\ensuremath{\hbar}}
\newcommand{\islow}{\ensuremath{low?}}


%%%%%%%%%%%%%%%%%%%%%%%%                  some macros             %%%%%%%%%%%%%%%%%%%%%%%%%%%%

% Proofs

% For proofs% PROOF ENVIRONMENT (from PT)
% it puts a box (or whatever \markendofproof) at the end of each
% proof, unless the author has already done so with the command \qed.
 
% the ``proof'' environment itself

% \newcommand{\scfont}[1]{{\sc #1}} % renew this for latex2e
\def\proof{\followon{Proof}}
\def\endproof{\ifSuppressEndOfProof\global\SuppressEndOfProoffalse
   \else\xqed\fi\endfollowon}
\def\followon#1{\trivlist\item[\hskip\labelsep{\textbf{#1}.}]}
\def\endfollowon{\endtrivlist}

% command to push box to right of page
\def\pushright#1{{\parfillskip=0pt\widowpenalty=10000
\displaywidowpenalty=10000\finalhyphendemerits=0\leavevmode\unskip
\nobreak\hfil\penalty50\hskip.2em\null\hfill{#1}\par}}

% command to force early end of proof marker. This is used if the end
% of proof marker would otherwise come out in the wrong place.
\def\qed{\xqed\global\SuppressEndOfProoftrue}
\newif\ifSuppressEndOfProof\SuppressEndOfProoffalse
\def\xqed{\pushright\markendofproof}
\def\markendofproof{$\Box$} % rule{1.3217ex}{2ex}}


%\newcommand{\proofbeg}{\par{\bf Proof. }}
%\newcommand{\proofsketch}{\par{\bf Proof sketch. }}
%\newcommand{\proofend}{\mbox{$\Box$}}

\newcommand{\corol}{\par{\bf Corollary. }}

\newcommand{\ex}{\par{\bf Example. }}

\newcommand{\draft}{{\bf \fbox{DRAFT}}}

\newtheorem{Theorem}{Theorem}
\newtheorem{Proposition}{Proposition}
\newtheorem{Corollary}{Corollary}
\newtheorem{Lemma}{Lemma}
\newtheorem{Definition}{Definition}
\newtheorem{Observation}{Observation}

\newcommand{\nb}[1]{$\blacktriangleleft$\ {\bf #1\/} $\blacktriangleright$\ 
\marginpar{{\bf NB}}}                     % for visible comments;
%\newcommand{\nb}[1]{}                   % select this line to hide them

\newcommand{\nat}{\mathbb{N}} 
%\newcommand{\nat}{{\boldmath N}} 
%\newcommand{\equals}{\stackrel{.}{=}}

%\newcommand{\para}[2][]{\prod_{#1} #2} 
%\newcommand{\para}[2][]{\! <\! #2\! >_{#1}\!} 
\newcommand{\para}[1]{\langle #1 \rangle}
\newcommand{\paral}{\mid}

% experiments to tame the latex8 sf

%\renewcommand{\sfdefault}{phv,10pt}

%  add the period after section numbers 

%\renewcommand{\section}[1]{\Section{\hskip -1em.~#1}} 
%\renewcommand{\subsection}[1]{\SubSection{\hskip -1em.~#1}}
                                       
%% a hack for latex8 style

%\newfont{\emailfont}{phv12}

\newcommand{\codcaption}[1]{\mathtt{#1}}
%\newcommand{\codemail}[1]{\textsf{\normalsize #1}}
%\newcommand{\codfoot}[1]{{\fontfamily{phv}\fontsize{7}{.3cm}\selectfont #1}}
%\newcommand{\codsubsec}[1]{\textsf{\normalsize #1}}

% from former dec-dave.tex

\newcommand{\cod}[1]{\mathtt{#1}} % program text   
                                          %% \small was 
                                          %% a hack for latex8 style
\newcommand{\true}{\cod{True}}
\newcommand{\false}{\cod{False}}
\newcommand{\If}{\cod{if}}
\newcommand{\ifthenelse}[3]{\cod{if}~#1~\cod{then}~#2~\cod{else}~#3}
\newcommand{\ifthen}[2]{\cod{if}~#1~\cod{then}~#2}
\newcommand{\while}[2]{\cod{while}~#1~\cod{do}~#2}
\newcommand{\Int}{\mathit{Int}} % Type of Integers
\newcommand{\choice}{\talloblong}
\newcommand{\await}[2]{\cod{await}~#1~\cod{then}~#2}
\newcommand{\spawn}[1]{\cod{spawn}~#1}
\newcommand{\fork}[2]{\cod{fork}(#1,#2)}
\newcommand{\forkproc}[2]{\ensuremath{\lambda}#1.#2}
\newcommand{\forklocalvars}[3]{\cod{fork}((\forkproc{#1}{#2})\, @ \, #3)}

\newcommand{\out}[1]{\cod{output}(#1)}

\newcommand{\hforkbyvalue}[3]{\cod{hfork}_{[#1]}(#2,#3)}

\newcommand{\hide}{\cod{hide}}
\newcommand{\edih}{\cod{unhide}}
\newcommand{\yield}{\cod{yield}}
\newcommand{\hfork}[2]{\cod{hfork}(#1,#2)}
\newcommand{\Loop}{\cod{loop}}
\newcommand{\Skip}{\cod{skip}}
\newcommand{\Empty}{\emptyset}
\DeclareMathAccent{\arrowvec}{\mathord}{letters}{"7E}
\newcommand{\vect}[1]{\arrowvec{#1}}              % LaTeX 2.09 
\newcommand{\const}[1]{\mbox{\it #1}}
%\newcommand{\halt}{\cod{stop}}
\newcommand{\halt}{\mathit{stop}}
%\renewcommand{\vec}[1]{\error}   

\newcommand{\decl}[1]{\cod{declassify}(#1)}


% Synchronisation
\newcommand{\wait}[1]{\cod{P}(#1)}
\newcommand{\signal}[1]{\cod{V}(#1)}
\newcommand{\sem}{\mathit{s}}
\newcommand{\Sem}{\mathit{Sem}}
\newcommand{\newSem}[1]{\cod{newSem}(#1)}
\newcommand{\gSem}{\mathit{w}}   % Global semaphore in a thread
\newcommand{\nSem}{\mathit{s}}   % Global semaphore in a thread
\newcommand{\bSem}{\mathit{b}}   % Global semaphore in a thread

\newcommand{\va}[1]{\mathit{#1}} % to be used for variables
\newcommand{\comment}[1]{\cod{/*}\ \text{#1}\ \cod{*/}}


%\newcommand{\blocked}[2]{\cod{blocked}(#1,#2)}
%\newcommand{\signalled}[2]{\cod{signalled}(#1,#2)}
%\newcommand{\labblocked}{\mathit{blocked}}
%\newcommand{\labsignalled}{\mathit{signalled}}
\newcommand{\labblocked}[1]{\otimes#1}
\newcommand{\labsignalled}[1]{\odot#1}

\newcommand{\syncConfig}[3]{\llocal #1,#2,#3 \rlocal}

\newcommand{\Config}[4]{\llocal \para{#1},#2,#3,#4 \rlocal}

% Semantics

%\newcommand{\union}{\mathord{\cup}}
\newcommand{\Mod}{\mathbin{\mathbf{mod}}}

\newcommand{\term}{\downarrow}

\newcommand{\configs}{\mathit{Config}}
\newcommand{\commands}{\mathit{Com}}
\newcommand{\pools}{\vect{\commands}}


\newcommand{\config}[2]{\langle #1,#2 \rangle}

\newcommand{\llocal}{\langle\hspace{-2.4 pt}|}
\newcommand{\rlocal}{|\hspace{-2.4 pt}\rangle}
%\newcommand{\config}[3]{\langle #1,#2,#3 \rangle}
%\newcommand{\configF}[2]{\langle #1,#2 \rangle}
\newcommand{\configF}[1]{\langle \halt,#1 \rangle}

\newcommand{\configE}[3]{\langle #1,#2,#3 \rangle}
\newcommand{\configEF}[2]{\langle #1,#2 \rangle}

\newcommand{\emptyConfig}{\para{}}
\newcommand{\emptyVec}{\para{}}

\newcommand{\detarrow}{\rightarrowtriangle}
%\newcommand{\labarrow}[1]{\stackrel{#1}{\longrightarrow}}
%\newcommand{\labarrow}[1]{\stackrel{#1}{\rightarrowtriangle}}
\newcommand{\probarrow}[1]{\rightarrow_{#1}}
\newcommand{\parrow}{\rightarrow}
\newcommand{\concarrow}{\rightarrow}

\newcommand{\psetarrow}[1]{\probarrow{#1}}     % the same as \probarrow for the moment

\newcommand{\thread}[1]{\ensuremath{c_{#1}}}
\newcommand{\threadp}[1]{\ensuremath{c'_{#1}}}

\newcommand{\nthreads}
{
 \thread{1}
 \dots
 \thread{n}
}

\newcommand{\selectthread}[4]
{
 \thread{1}
 \dots
 {#1}
 {#2}
 {#4}
 {#3}
 \dots
 \thread{n}
}


\newcommand{\nthreadsp}
{
 \threadp{1}{f_1}
 \dots
 \threadp{n}{f_n}
}

\newcommand{\create}{\leadsto}
\newcommand{\createClose}{\leadsto^*}

\newcommand{\newthreads}{{\vect{d}}} 
\newcommand{\namethreads}[1]{N(#1)}


\newcommand{\cnt}{\textrm{cnt}} % count field of semaphore
\newcommand{\que}{\textrm{que}} %  queue field of semaphore

% Memory / Lock Sets

\newcommand{\mem}{m}
\newcommand{\lockset}{\omega}
\newcommand{\hiddenthreads}{\phi}

% Schedulers semantic

\newcommand{\Schdlconfig}[2]
{\llocal #1, #2 \rlocal}

\newcommand{\Schdlresult}[2]
{\llocal #1, #2 \rlocal}

\newcommand{\hidearrow}[1]{\labarrow{#1 \leadsto \bullet}}
\newcommand{\ediharrow}[1]{\labarrow{\bullet \leadsto #1}}
\newcommand{\yieldarrow}{\labarrow{\leadsto \!\!\!\!/}}
\newcommand{\hiddenforkarrow}{\labarrow{\bullet}}
\newcommand{\hiddenforkarrowvect}[1]{\labarrow{\bullet_{\vect{#1}}}}
\newcommand{\hiddenforkarrowvectbyvalue}[2]{\labarrow{\bullet_{\vect{#1}}\ ,\  #2}}
\newcommand{\forkarrow}{\labarrow{\circ}}
\newcommand{\forkarrowvect}[1]{\labarrow{\circ_{\vect{#1}}}}
\newcommand{\nonfinishedstep}[1]{\labarrow{#1 \, \leadsto}} 
\newcommand{\finishedstep}[1]{\labarrow{#1 \, \leadsto \times}} 
\newcommand{\pickupthread}[1]{\labarrow{\uparrow {#1}}} 
\newcommand{\pickuphighthread}[1]{\labarrow{\uparrow_s{#1}}} 
\newcommand{\choosescheduler}[2]{\labarrow{\otimes{#1}{#2}}} 
\newcommand{\lockedthread}[2]{\labarrow{\labblocked{#1}_{#2}}} 
\newcommand{\signaledthread}[2]{\labarrow{\labsignalled{#1}_{#2}}}

\newcommand{\multiconfig}[4]
{ 
  {\llocal #1,#2,#3,#4 \rlocal}
}

% Schedulers

%\newcommand{\hist}{\Xi}  % too squiggly
\newcommand{\hist}{Hist}
\newcommand{\emptyHist}{\epsilon}
\newcommand{\live}[1]{live(#1)}

\newcommand{\schedule}{\sigma}
\newcommand{\scheduleMem}{\eta}

% Types etc

\newcommand{\high}{\mathit{H}}
\newcommand{\low}{\mathit{L}}
\newcommand{\cmd}[1]{\mathit{#1\ cmd}}
\newcommand{\var}[1]{\mathit{#1\ var}}
\newcommand{\To}{\rightarrow} % function space / type arrow

% Bisimulations

\newcommand{\peq}{\sim_L}
\newcommand{\pseq}[1]{\sim_L^{#1}}
\newcommand{\psieq}{\approx_L}
\newcommand{\pstrong}{\approxeq_L}

\newcommand{\possim}{\preceq}
\newcommand{\posbisim}{\simeq}



% Distribution

\newcommand{\distr}[1]{{\cal D}(#1)}

% --- PERS etc-----------------------
%\newcommand{\Rel}[1]{\mathit{Rel}(#1)} % Relations on A 
\newcommand{\Per}[1]{\mathit{Per}(#1)} % Pers on A
\newcommand{\Domain}[1]{\mathit{dom}(#1)}          % The domain of a relation
\newcommand{\Ker}[1]{\mathit{ker}(#1)} % The kernal of a function
\newcommand{\All}{\mathit{All}}
\newcommand{\Id}{\mathit{Id}}
\newcommand{\Permap}{\rightarrowtriangle} % the per arrow (exponent)

% Domains etc 
\newcommand{\State}{\mathbf{St}}


% Logical ops (for use in Math mode)
\newcommand{\Implies}{\Longrightarrow}
\newcommand{\isimpliedby}{\Leftarrow}
\newcommand{\Iff}{\Longleftrightarrow}


% iff is predefined % \newcommand{\iff}{\Leftrightarrow}

\newcommand{\isbydef}{\stackrel{\mbox{\tiny def}}{=}}
\newcommand{\AND}{\mathrel{\wedge}}
\newcommand{\OR}{\mathrel{\vee}}
% \newcommand{\Subset}{\subseteq}
\newcommand{\Powerset}{\wp}
\newcommand{\Set}[1] % gives a set containing its arg.
           {\left\{#1\right\}}


\newcommand{\lmulti}{\{\hspace{-2.5 pt}|}
\newcommand{\rmulti}{|\hspace{-2.5 pt}\}}
\newcommand{\multiset}[1] % gives a set containing its arg.
           {\lmulti#1\rmulti}

\newcommand{\Setminus}{\setminus}
\newcommand{\SetDef}[2] % {x}{y} => { x | y }
           {\left\{#1\mathrel{\mid}#2\right\}}
\newcommand{\Union} {\cup}
\newcommand{\Intersect} {\cap}
\newcommand{\ForAll}[2]
           {\forall #1 .\,#2}
\newcommand{\Exists}[2]
           {\exists #1 .\,#2}
\newcommand{\entails}{\vdash}
%\newcommand{\Nat}{I\!\!N}

% SIMPLE INFERENCE RULES

% added an optional parameter (a side condition)
\newcommand{\Infer}[3] %  name X top X bottom, ie
                       %    #2
                       % #1---- 
                       %    #3
{\displaystyle \mbox{\footnotesize{#1}}
               \frac{\displaystyle{#2\vspace{.4cm}}}
                    {#3}
} 

% added an optional parameter (a side condition) for use in xalign
\newcommand{\Inferr}[3] %  name X top X bottom, ie
                       %        #2
                       % (#1)   ---- 
                       %        #3
{[#1]&&\displaystyle &\frac{\displaystyle{#2\vspace{.4cm}}}
                                         {#3}
}

\newcommand{\InferrA}[3] %  name X top X bottom, ie
                       %        #2
                       % (#1)   ---- 
                       %        #3
{[#1]\displaystyle &\frac{\displaystyle{#2\vspace{.4cm}}}
                                         {#3}
}

% added an optional parameter (a side condition) for use in xalign
% \newcommand{\Rule}[2]  %  name X top X bottom, ie
%                        %          
%                        % (#1)   #2
%                        %          
% {[#1]&&\displaystyle &\displaystyle{#2}
% }

% \newcommand{\RuleA}[2]  %  name X top X bottom, ie
%                        %          
%                        % (#1)   #2
%                        %          
% {[#1]&\displaystyle{#2}
% }

% Transformation to avoid timing leaks

\newcommand{\transto}{\hookrightarrow}
\newcommand{\transtohigh}{\hightransto}

% Misc

\newcommand{\eol}{\\[.8em]}
\newcommand{\Eol}{\\[1.5em]}

\newcommand{\sS}{{\scriptstyle\mathbb{S}}} % state of Skou's pbisimulation
\newcommand{\tT}{{\scriptstyle\mathbb{S}}'} % state of Skou's pbisimulation

%\newcommand{\sS}{s\hspace{-3.7 pt}s} % state of Skou's pbisimulation

% Formatting

%Compact headings
% \newcommand{\Subsection}[1]{\paragraph{\rm\textbf{#1.}}}
% \newcommand{\Paragraph}[1]{\textbf{#1}}
% \newcommand{\Term}[1]{\textbf{#1}}

%Full headings
%\newcommand{\Subsection}[1]{\subsection{#1}}
%\newcommand{\Paragraph}[1]{\paragraph{#1}}

\newcommand{\cfg}{\mathit{cfg}}
\newcommand{\cfgm}{\mathit{cfgm}}
\newcommand{\configTop}[3]{\langle   #2 ~ |_{#1} ~ #3  \rangle}

\newcommand{\stutterarrow}{\parrow^{\star}}

%%%%%%%%%%%%%%%%%%%%%%%%%%%%%%%%%%%%%%%%%%%%%%%%%%%%%%%%%%%%%%%%%%%%%
%Dec 2005
%%%%%%%%%%%%%%%%%%%%%%%%%%%%%%%%%%%%%%%%%%%%%%%%%%%%%%%%%%%%%%%%%%%%%

\newcommand{\G}{\Gamma}
\newcommand{\proves}{\ensuremath{\vdash}}

\newcommand{\barrier}{\mathit{barrier}}
\newcommand{\semP}[1]{\cod{p}(#1)}
\newcommand{\semV}[1]{\cod{v}(#1)}
%\newcommand{\lock}{\semP{\mathit{lock}}}
%\newcommand{\unlock}{\semV{\mathit{lock}}}
\newcommand{\lock}{\cod{lock}}
\newcommand{\unlock}{\cod{unlock}}
\newcommand{\hightransto}{\looparrowright}

\newcommand{\Rule}[2]{\infer{#2}{#1}} % this uses the proof package
\newcommand{\labRule}[3]{\infer[#1]{#3}{#2}} % this uses the proof package

\newcommand{\Vars}[1]{\mathit{Vars}(#1)}
\newcommand{\IDs}{\mathit{IDs}}
\newcommand{\Vals}{\mathit{Vals}}

\newcommand{\Rel}{\mathit{Rel}}

\newcommand{\configTwo}[2]{\langle #1,#2 \rangle}
\newcommand{\configThree}[3]{\langle #1,#2,#3 \rangle}

%\newcommand{\down}{\mathit{D}}
%\newcommand{\up}{\mathit{U}}

\newcommand{\vars}{\mathit{Vars}}

\newcommand{\inter}{\ensuremath{\cap}}
\newcommand{\union}{\ensuremath{\cup}}

\newcommand{\pc}{\mathit{pc}}
\newcommand{\join}{\sqcup}


%%%%%%%%%%%%%%%%%%%%%%%%%%%%%%%%%%%%%%%%%%%%%%%%%%%%%%%%%%%%%%%%%%%%%
%Dec 2005 - Ale
%%%%%%%%%%%%%%%%%%%%%%%%%%%%%%%%%%%%%%%%%%%%%%%%%%%%%%%%%%%%%%%%%%%%%
\if 0
\newcommand{\mat}[1]{\ensuremath{#1}}
\newcommand{\word}[1]{\textsl{#1}}
\newcommand{\Assgn}{\ensuremath{\Gamma}}
\newcommand{\AssgnOk}{\ensuremath{\Gamma_{\nu}}}
\newcommand{\ofType}[3]{\ensuremath{#1 \vdash #2 : #3}}
\newcommand{\FV}[1]{\word{FV}\mat{(#1)}}
\newcommand{\pc}{\word{pc}}
\newcommand{\hc}{\word{hc}}
\newcommand{\join}{\mat{\sqcup}}

\newcommand{\ND}{\mathit{ND}}
\newcommand{\RR}{\mathit{RR}}
\newcommand{\ws}{\mathit{ws}}
\fi

\newcommand{\monstack}{\mathit{st}}
%\newcommand{\configM}[2]{\langle #1,#2 \rangle}
\newcommand{\configM}[1]{#1}
\newcommand{\larrow}[1]{\stackrel{#1}{\longrightarrow}}
\newcommand{\nop}{\mathit{nop}}
\newcommand{\levelof}[1]{\mathit{lev}(#1)}
\newcommand{\labarrow}[2]{{\stackrel{#1}{\longrightarrow}}_{#2}}
\newcommand{\nolabarrow}{\longrightarrow}
\newcommand{\monitorevent}{\alpha}
\newcommand{\cend}{\mathit{end}}

\newcommand{\emptyStack}{\epsilon}


% Local Variables: 
% mode: latex
% TeX-master: "main"
% End: 
